\documentclass{article}

\usepackage[osf]{mathpazo}
\usepackage{microtype}

\usepackage{amsmath}
\usepackage{amssymb}
\usepackage{amsfonts}
\usepackage{amsthm}
\usepackage{mathtools}
\usepackage{bm}
\usepackage{xcolor}
\usepackage{graphicx}
\usepackage{tikz}
\usepackage{enumitem}
\usepackage[alsoload=binary]{siunitx}
\usepackage{booktabs}
\usepackage{multirow}
\usepackage{fancyvrb}
\usepackage[section]{placeins}
\usepackage{flafter}
\usepackage[backgroundcolor=orange!30,bordercolor=none]{todonotes}
\usepackage{pdfpages}
\usepackage[hmargin=1in]{geometry}
\usepackage{url}
\usepackage{hyperref}

% no section numbers
\setcounter{secnumdepth}{-2}
\pagestyle{empty}

% a bit more compact, and fixes spacing issues
\let\originalleft\left
\let\originalright\right
\renewcommand{\l}{\mathopen{}\mathclose\bgroup\originalleft}
\renewcommand{\r}{\aftergroup\egroup\originalright}

% commands
\renewcommand\vec[1]{\bm{#1}}
\renewcommand\epsilon{\varepsilon}
\newcommand\bit{\ensuremath{\set{0,1}}}
\newcommand\abs[1]{\l\vert #1 \r\vert}
\newcommand\set[1]{\l{ #1 \r}}

% write in code
\DefineShortVerb{\|}

\usetikzlibrary
{
  positioning,
  calc,
  intersections,
  arrows,
  shapes.misc,
}

% redefine `\to` to use tikz arrows
\renewcommand\to{%
  \mathbin{\tikz[baseline=-.7ex] \draw[->] (0,0) -- +(.4,0);}%
}

\tikzset
  {
    ->,
    >=latex',
    shorten >=.5pt,
    node distance=.5cm and .7cm,
    auto,
    wide fork/.style= {node distance=.5cm and 1.2cm},
    block/.style=
      {
        draw,
        inner sep=.3em,
        rounded rectangle,
        minimum size=1.7em,
      },
    terminal/.style=
      {
        block,
        rectangle,
        text height=1.5ex,
        text depth=.25ex,
      },
    inline center/.style={baseline=-.5ex},
    retreating/.style={dashed},
    cross/.style={dashdotted},
    forward/.style={dotted},
  }

\begin{document}

\includepdf
  [
    pages=1,
    pagecommand=
      {
        \begin{tikzpicture}[remember picture, overlay]
          \node[text width=6.2in, text height=2.3in]
          at (current page.north east) {\large Sam Britt};
        \end{tikzpicture}
      }
  ]
  {Assignment2ProblemSet1}

\noindent
Sam Britt     \hfill CS 6340      \\
Jan. 19, 2013 \hfill Assignment 2 \\

\begin{enumerate}
  \item The CFGs are below. A join node, J, has been added for
    clarity.

    \begin{minipage}[t]{\linewidth}
      \begin{minipage}[t]{.5\linewidth}
        \centering
        \begin{minipage}[t]{.6\linewidth}
          \begin{center}
            \begin{tikzpicture}
              \node[terminal]                   (E) {entry, 1, 2};
              \node[block,below=of E]           (3) {3};
              \node[block,below left=of 3]      (4) {4, 5};
              \node[block,below left=of 4]      (7) {7, 8};
              \node[terminal,below right=of 4] (12) {12, exit};
              \node[block,below left=of 7]     (10) {10};
              \node[block,below right=of 10]    (J) {J};

              \path (E) edge                           (3);
              \path (3) edge            node[swap] {T} (4)
                        edge[bend left] node       {F} (12);
              \path (4) edge            node[swap] {F} (7)
                        edge            node       {T} (12);
              \path (7) edge            node[swap] {F} (10)
                        edge[bend left] node       {T} (J);
              \path (10) edge                          (J);

              \node[node distance=.3cm,left=of 10,coordinate] (a) {};
              \draw (J) -| (a) |- (3);
            \end{tikzpicture}
          \end{center}
          CFG using maximal basic blocks.
        \end{minipage}
      \end{minipage}
      %
      \begin{minipage}[t]{.5\linewidth}
        \centering
        \begin{minipage}[t]{.6\linewidth}
          \begin{center}
            \begin{tikzpicture}
              \node[terminal]               (E) {entry};
              \node[block,below=of E]       (1)  {1};
              \node[block,below=of 1]       (2)  {2};
              \node[block,below=of 2]       (3)  {3};
              \node[block,below left=of 3]  (4)  {4};
              \node[block,below=of 4]       (5)  {5};
              \node[block,below left=of 5]  (7)  {7};
              \node[block,below right=of 5](12) {12};
              \node[block,below=of 7]       (8)  {8};
              \node[block,below left=of 8] (10) {10};
              \node[block,below right=of 10](J)  {J};
              \node[terminal,below=of 12]   (X) {exit};

              \path (E) edge (1);
              \path (1) edge (2);
              \path (2) edge (3);
              \path (3) edge node[swap] {T} (4)
                        edge[bend left] node {F} (12);
              \path (12) edge (X);
              \path (4) edge (5);
              \path (5) edge node[swap] {F} (7)
                        edge node {T} (12);
              \path (7) edge (8);
              \path (8) edge node[swap] {F} (10)
                        edge[bend left] node {T} (J);
              \path (10) edge (J);

              \node[node distance=.3cm,left=of 10,coordinate] (a) {};
              \draw (J) -| (a) |- (3);
            \end{tikzpicture}
          \end{center}
          CFG using one statement per block.
        \end{minipage}
      \end{minipage}
    \end{minipage}

    \newpage
  \item
    \begin{enumerate}
      \item The dominator and postdominator trees are below.

        \begin{minipage}[t]{\linewidth}
          \begin{minipage}[t]{.5\linewidth}
            \centering
            \begin{minipage}[t]{.4\linewidth}
              \begin{center}
                \begin{tikzpicture}
                  \node[terminal]               (E) {entry};
                  \node[block,below=of E]       (1)  {1};
                  \node[block,below=of 1]       (2)  {2};
                  \node[block,below=of 2]       (3)  {3};
                  \node[block,below left=of 3]  (4)  {4};
                  \node[block,below=of 4]       (5)  {5};
                  \node[block,below=of 5]       (7)  {7};
                  \node[block,below=of 7]       (8)  {8};
                  \node[block,below=of 8]      (10) {10};
                  \node[block,below right=of 3](12) {12};
                  \node[terminal,below=of 12]   (X) {exit};

                  \path (E) edge (1);
                  \path (1) edge (2);
                  \path (2) edge (3);
                  \path (3) edge (4) edge (12);
                  \path (4) edge (5);
                  \path (5) edge (7);
                  \path (7) edge (8);
                  \path (8) edge (10);
                  \path (12) edge (X);
                \end{tikzpicture}
              \end{center}
              Dominator tree.
            \end{minipage}
          \end{minipage}
          %
          \begin{minipage}[t]{.5\linewidth}
            \centering
            \begin{minipage}[t]{.8\linewidth}
              \begin{center}
                \begin{tikzpicture}
                  \node[terminal]               (X) {exit};
                  \node[block,below=of X]      (12) {12};
                  \node[block,wide fork,below left=of 12] (5)  {5};
                  \node[block,below right=of 5]       (4)  {4};
                  \node[block,below  left=of 5]       (7)  {7};
                  \node[block,wide fork,below right=of 12](3)  {3};

                  \node[block,below left=of 3] (10) {10};
                  \node[block,below=of 3]       (8)  {8};
                  \node[block,below right=of 3] (2)  {2};
                  \node[block,below=of 2]       (1)  {1};
                  \node[terminal,below=of 1]    (E)  {entry};

                  \path (X) edge (12);
                  \path (12) edge (5) edge (3);
                  \path (5) edge (7) edge (4);
                  \path (3) edge (10) edge (8) edge (2);
                  \path (2) edge (1);
                  \path (1) edge (E);

                \end{tikzpicture}
              \end{center}
              Postdominator tree.
            \end{minipage}
          \end{minipage}
        \end{minipage}

      \item Yes, the graph is reducible. First, many nodes have a
        single predecessor, so T2 analysis combines many of the nodes:

        \begin{center}
          \begin{tikzpicture}
            \node[terminal]               (E) {entry, 1, 2};
            \node[block,below=of E]       (3) {3};
            \node[block,below  left=of 3] (4) {4--8, 10};
            \node[block,below right=of 3](12) {12, exit};

            \path (E) edge (3);
            \path (3) edge (4)
                      edge (12);
            \path (4) edge (12);

            \node[node distance=.3cm,left=of 4,coordinate] (a) {};
            \draw (4) -| (a) |- (3);
          \end{tikzpicture}
        \end{center}
        Again applying a T2 transformation:
        \begin{center}
          \begin{tikzpicture}
            \node[terminal]               (E) {entry, 1, 2};
            \node[block,below=of E]       (3) {3--8, 10};
            \node[block,below right=of 3](12) {12, exit};

            \path (E) edge (3);
            \path (3) edge (12);
            \path (3) edge[loop left] (3);
          \end{tikzpicture}
        \end{center}
        Applying a T1 transformation to get rid of the self loop and a
        T2 transformation to collapse the node containing the exit
        statement leaves:
        \begin{center}
          \begin{tikzpicture}
            \node[terminal]               (E) {entry, 1, 2};
            \node[block,below=of E]       (3) {3--8, 10, 12, exit};

            \path (E) edge (3);
          \end{tikzpicture}
        \end{center}
        And finally, a T1 transformation collapses this flow graph to
        a single node, thus showing that the original CFG is indeed
        reducible.
        \begin{center}
          \begin{tikzpicture}
            \node[terminal, draw] (E) {entry, 1--8, 10, 12, exit};
          \end{tikzpicture}
        \end{center}

      \item One version of the dept-first presentation is below.
        Retreating edges are marked by dashes
        (\tikz[inline center] \draw[retreating] (0,0) -- (.7,0);)
        and forward edges are marked by dots
        (\tikz[inline center] \draw[forward] (0,0) -- (.7,0);).
        The depth of this graph is $1$. The paths $10 \to 3$ or $8 \to
        3$ are both acyclic paths, each containing one reverse edge.
        Another path with more reverse edges cannot be found. (In
        fact, because both back edges have the same head node, they
        are in the same loop.)

        \begin{center}
          \begin{minipage}[t]{.53\linewidth}
            \begin{center}
              \begin{tikzpicture}[on grid, node distance=1.2cm]
                \node[terminal]                 (E)  {entry};
                \node[block,below left=of E]    (1)  {1};
                \node[block,below left=of 1]    (2)  {2};
                \node[block,below left=of 2]    (3)  {3};
                \node[block,below left=of 3]    (4)  {4};
                \node[block,below left=of 4]    (5)  {5};
                \node[block,below left=of 5]    (7)  {7};
                \node[block,below left=of 7]    (8)  {8};
                \node[block,below left=of 8]   (10) {10};

                \node[block,below right=of 5]  (12) {12};
                \node[block,below=of 12]        (X) {exit};

                \draw (E) -- (1);
                \draw (1) -- (2);
                \draw (2) -- (3);
                \draw (3) -- (4);
                \draw (4) -- (5);
                \draw (5) -- (7);
                \draw (7) -- (8);
                \draw (8) -- (10);
                \draw (5) -- (12);
                \draw (12) -- (X);

                \draw[retreating, name path=10--3] (10) |- (3);
                \path[name path=8--3] (8) |- (3);
                \path[name intersections={of=10--3 and 8--3, by=inter}];
                \draw[retreating,-] (8) -- (inter);

                \draw[forward] (3) |- (12);
              \end{tikzpicture}
            \end{center}
          Dept-first presentation.
        \end{minipage}
      \end{center}

    \end{enumerate}

  \newpage
  \item The CFGs are below. A join node, J, has been added for
    clarity.

    \begin{minipage}[t]{\linewidth}
      \begin{minipage}[t]{.45\linewidth}
        \centering
        \begin{minipage}[t]{\linewidth}
          \begin{center}
            \begin{tikzpicture}
              \node[terminal]                  (E) {entry, 1--4};
              \node[block,below right=of E]    (8) {8};
              \node[block,below  left=of E]    (5) {5, 6};
              \node[block,below  left=of 8]   (12) {12, 13};
              \node[block,below right=of 8]    (9) {9, 10};
              \node[block,below right=of 12]  (15) {15};
              \node[block,below  left=of 15]  (23) {23};
              \node[terminal,below left=of 23] (X) {exit};
              \node[block,below right=of 15]  (16) {16, 17};
              \node[block,below  left=of 16]  (20) {20};
              \node[block,below right=of 16]  (18) {18};
              \node (tmp)  at ($(20)!.5!(18)$) {};
              \node[block,below =of tmp]   (J) {J};

              \path (E) edge node[swap] {T} (5)
                        edge node {F} (8);
              \path (8) edge node[swap] {F} (12)
                        edge node {T} (9);
              \path (9) edge (15);
              \path (12) edge (15);
              \path (15) edge node[swap] {F} (23)
                         edge node {T} (16);
              \path (23) edge (X);
              \path (16) edge node[swap] {F} (20)
                         edge node {T} (18);
              \path (18) edge (J);
              \path (20) edge (J);

              \draw (5) |- (X);

              \node[node distance=.3cm,right=of 18,coordinate] (a) {};
              \draw (J) -| (a) |- (15);
            \end{tikzpicture}
          \end{center}
          CFG using maximal basic blocks.
        \end{minipage}
      \end{minipage}
      %
      \begin{minipage}[t]{.5\linewidth}
        \centering
        \begin{minipage}[t]{.6\linewidth}
          \begin{center}
            \begin{tikzpicture}
              \node[terminal] (E)                   {entry};
              \node[block,below=of E]          (1)  {1};
              \node[block,below=of 1]          (2)  {2};
              \node[block,below=of 2]          (3)  {3};
              \node[block,below=of 3]          (4)  {4};
              \node[block,below right=of 4]    (8)  {8};
              \node[block,below  left=of 4]    (5)  {5};
              \node[block,below  left=of 5]    (6)  {6};
              \node[block,below  left=of 8]   (12) {12};
              \node[block,below=of 12]        (13) {13};
              \node[block,below right=of 8]    (9)  {9};
              \node[block,below=of 9]         (10) {10};
              \node[block,below right=of 13]  (15) {15};
              \node[block,below  left=of 15]  (23) {23};
              \node[terminal,below=of 23]      (X) {exit};
              \node[block,below right=of 15]  (16) {16};
              \node[block,below=of 16]        (17) {17};
              \node[block,below  left=of 17]  (20) {20};
              \node[block,below right=of 17]  (18) {18};
              \node[block,below right=of 20]   (J)  {J};

              \path (E) edge (1);
              \path (1) edge (2);
              \path (2) edge (3);
              \path (3) edge (4);
              \path (4) edge node[swap] {T} (5)
                        edge node {F} (8);
              \path (5) edge (6);
              \path (8) edge node[swap] {F} (12)
                        edge node {T} (9);
              \path (9) edge (10);
              \path (10) edge (15);
              \path (12) edge (13);
              \path (13) edge (15);
              \path (15) edge node[swap] {F} (23)
                         edge node {T} (16);
              \path (23) edge (X);
              \path (16) edge (17);
              \path (17) edge node[swap] {F} (20)
                         edge node {T} (18);
              \path (18) edge (J);
              \path (20) edge (J);

              \draw (6) |- (X);

              \node[node distance=.3cm,right=of 18,coordinate] (a) {};
              \draw (J) -| (a) |- (15);
            \end{tikzpicture}
          \end{center}
          CFG using one statement per block.
        \end{minipage}
      \end{minipage}
    \end{minipage}
\end{enumerate}

\end{document}
