% vim: nowrap:guioptions+=b

\documentclass{article}

\usepackage[osf]{mathpazo}
\usepackage{microtype}

\usepackage{amsmath}
\usepackage{amssymb}
\usepackage{amsfonts}
\usepackage{amsthm}
\usepackage{mathtools}
\usepackage{bm}
\usepackage{xcolor}
\usepackage{graphicx}
\usepackage[pdftex]{lscape}
\usepackage{lipsum}
\usepackage{tikz}
\usepackage{enumitem}
\usepackage[alsoload=binary]{siunitx}
\usepackage{booktabs}
\usepackage{multirow}
\usepackage{tabularx}
\usepackage{fancyvrb}
\usepackage[section]{placeins}
\usepackage{flafter}
\usepackage{pdfpages}
\usepackage[hmargin=1.5in,vmargin=1in]{geometry}
\usepackage{url}
\usepackage{hyperref}

% algorithms
\let\oldgets\gets
\usepackage{clrscode3e}
\let\gets\oldgets

% no section numbers
\setcounter{secnumdepth}{-2}

% a bit more compact, and fixes spacing issues
\let\originalleft\left
\let\originalright\right
\renewcommand{\l}{\mathopen{}\mathclose\bgroup\originalleft}
\renewcommand{\r}{\aftergroup\egroup\originalright}

% commands
\renewcommand\vec[1]{\bm{#1}}
\renewcommand\epsilon{\varepsilon}
\newcommand\bit{\ensuremath{\set{0,1}}}
\newcommand\abs[1]{\l\vert #1 \r\vert}
\newcommand\set[1]{\ensuremath{\l\{#1\r\}}}

% red todos
\usepackage{marginnote}
\renewcommand\marginfont{\color{red}}
\newcommand\todo[1]{{\color{red}*}\marginnote{* #1}}

% write in code
\DefineShortVerb{\|}

\usetikzlibrary
{
  positioning,
  calc,
  intersections,
  arrows,
  shapes.misc,
  shapes.geometric,
}

% redefine `\to` to use tikz arrows
\renewcommand\to{%
  \mathbin{\tikz[baseline=-.7ex] \draw[->] (0,0) -- +(.4,0);}%
}

\tikzset
  {
    ->,
    >=latex',
    shorten >=.5pt,
    node distance=.5cm and .7cm,
    auto,
    fork/.style= {node distance=.5cm and #1},
    wide fork/.style={fork=1.2cm},
    block/.style=
      {
        draw,
        inner sep=.3em,
        rounded rectangle,
        minimum size=1.7em,
      },
    terminal/.style=
      {
        block,
        rectangle,
        text height=1.5ex,
        text depth=.25ex,
      },
    inline center/.style={baseline=-.5ex},
    retreating/.style={dashed},
    cross/.style={dashdotted},
    forward/.style={dotted},
  }

% table customizaton
\newcommand\newrow{\\\addlinespace}

% center columns
\newcolumntype{M}[1]{>{\centering\arraybackslash}m{#1}}

% set
\newcommand\s[1]{\{#1\}}

% highlight
\newcommand\hi[1]{\textcolor{red}{#1}}

\sisetup{math-rm=\rmfamily}

\begin{document}
\thispagestyle{empty}

\includepdf
  [
    pages=1,
    pagecommand=
      {
        \begin{tikzpicture}[remember picture, overlay]
          \node[text width=6.3in, text height=2.6in]
          at (current page.north east)
          {\large
            Sam Britt, Shriram Swaminathan,\\[1.5em]
            and Sivaramachanran Ganesan
          };
        \end{tikzpicture}
      }
  ]
  {Assignment3ProblemSet2}

\clearpage
\pagenumbering{arabic}

\noindent
Sam Britt               \hfill CS 6340      \\
Shriram Swaminathan     \hfill Assignment 3 \\
Sivaramachanran Ganesan \hfill Jan. 26, 2013

\begin{enumerate}
  \item
    \begin{enumerate}
      \item Reaching definitions are shown in Table~\ref{reaching-defs}.
        Although the data is
        presented in order of increasing block number, the table was
        actually created in the following depth-first order:
        1, 2, 3, 4, 8, 12, 13, 9, 10, 15, 23, 16, 17, 20, 18, 5, 6.

      \item Reachable uses are shown in Table~\ref{reachable-uses}.
        Although the data is presented in order of increasing block
        number, the table was actually created in the following
        reversed depth-first order:
        6, 5, 18, 20, 17, 16, 23, 15, 10, 9, 13, 12, 8, 4, 3, 2, 1.

      \item Definition-use pairs are shown in
        Table~\ref{def-use-pairs}, grouped according to use.
    \end{enumerate}

  \item
    \begin{enumerate}
      \item First, we create the augmented CFG and the postdominator
        tree:

        \begin{minipage}[t]{\linewidth}
          \begin{minipage}[t]{.5\linewidth}
            \centering
            \begin{minipage}[t]{.9\linewidth}
              \begin{center}
                \begin{tikzpicture}
                  \node[terminal]               (E) {entry};
                  \node[block,below=of E]       (1)  {1};
                  \node[block,below=of 1]       (2)  {2};
                  \node[block,below=of 2]       (3)  {3};
                  \node[block,below left=of 3]  (4)  {4};
                  \node[block,below=of 4]       (5)  {5};
                  \node[block,fork=.8cm,below left=of 5]  (7)  {7};
                  \node[block,fork=.8cm,below right=of 5] (6)  {6};
                  \node[block,below=of 6]      (12) {12};
                  \node[block,below=of 7]       (8)  {8};
                  \node[block,below left=of 8] (10) {10};
                  \node[block,below right=of 8] (9)  {9};
                  \node[block,below right=of 10](J)  {J};
                  \node[terminal,below=of 12]   (X) {exit};

                  \path (E) -- node (mid) {} (X);
                  \node[terminal,node distance=1cm,right=of mid] (S) {start};

                  \path (E) edge (1);
                  \path (1) edge (2);
                  \path (2) edge (3);
                  \path (3) edge node[swap] {T} (4)
                            edge[bend left] node {F} (12);
                  \path (12) edge (X);
                  \path (4) edge (5);
                  \path (5) edge node[swap] {F} (7)
                            edge node {T} (6);
                  \path (6) edge (12);
                  \path (7) edge (8);
                  \path (8) edge node[swap] {F} (10)
                            edge node {T} (9);
                  \path (10) edge (J);
                  \path (9) edge (J);

                  \node[node distance=.3cm,left=of 10,coordinate] (a) {};
                  \draw (J) -| (a) |- (3);

                  \draw (S) |- node[right,near start] {T} (E);
                  \draw (S) |- node[right,near start] {F} (X);
                \end{tikzpicture}
              \end{center}
              Augmented CFG.
            \end{minipage}
          \end{minipage}
          %
          \begin{minipage}[t]{.5\linewidth}
            \centering
            \begin{minipage}[t]{.9\linewidth}
              \begin{center}
                \begin{tikzpicture}
                  \node[terminal]                             (X)  {exit};
                  \node[terminal,below right=of X]            (S)  {start};
                  \node[block,below=of X]                    (12)  {12};
                  \node[block,wide fork,below left=of 12]     (3)  {3};
                  \node[block,wide fork,below right=of 12]    (5)  {5};
                  \node[block,below=of 12]                    (6)  {6};
                  \node[block,below=of 5]                     (4)  {4};
                  \node[block,fork=.4cm,below left=of 3]      (8)  {8};
                  \node[block,fork=.4cm,below right=of 3]     (9)  {9};
                  \node[block,node distance=.4cm,left=of 8]   (2)  {2};
                  \node[block,node distance=.4cm,right=of 9] (10) {10};
                  \node[block,below=of 2]                     (1)  {1};
                  \node[terminal,below=of 1]                  (E)  {entry};
                  \node[block,below=of 8]                     (7)  {7};

                  \path (X) edge (12);
                  \path (X) edge (S);
                  \path (12) edge (3) edge (5) edge (6);
                  \path (5) edge (4);
                  \path (3) edge (2) edge (8) edge (9) edge (10);
                  \path (2) edge (1);
                  \path (1) edge (E);
                  \path (8) edge (7);

                \end{tikzpicture}
              \end{center}
              Corresponding postdominator tree.
            \end{minipage}
          \end{minipage}
        \end{minipage}

        We then find the set $S$, where each element in $S$ is an edge
        $(A, B) \in G$, where $G$ is the above CFG and $B$ does not
        postdominate $A$. We find that
        \begin{equation*}
          S = \text{\s{(start, entry), (3, 4), (5, 6), (5, 7), (8, 9),
          (8, 10)}}.
        \end{equation*}
        For each edge $(A,B) \in S$, we find $L$ to be the common
        ancestor of $A$ and $B$. Finally, the nodes that are
        control-dependent on $A$ are those the path from $L$ to $B$ on
        the postdominator tree, including $B$, and including $L$ only
        if $L=A$. These results are summarized in the following table:
        \begin{center}
          \begin{tabular}{cccl}
            \toprule
            $(A,B) \in S$ & Condition (T/F) & $L$ & Nodes dependent on
            $A$ \\
            \midrule
            (start, entry) & T & exit & \s{entry, 1, 2, 3, 12} \\
            (3, 4)         & T & 12   & \s{4, 5} \\
            (5, 6)         & T & 12   & \s{6} \\
            (5, 7)         & F & 12   & \s{3, 7, 8} \\
            (8, 9)         & T & 3    & \s{9} \\
            (8, 10)        & F & 3    & \s{10} \\
            \bottomrule
          \end{tabular}
        \end{center}
        The control dependence graph is constructed directly from the
        above table.

        \begin{minipage}{\linewidth}
          \centering
          \begin{minipage}{.6\linewidth}
            \begin{center}
              \begin{tikzpicture}
                [
                  node distance=.8cm and 1cm,
                ]
                \node[terminal] (S) {start};
                \node[block, below=of S] (2) {2};
                \node[block, left=of 2]  (1) {1};
                \node[terminal, left=of 1]  (E) {entry};
                \node[block, right=of 2]  (3) {3};
                \node[block, right=of 3]  (12) {12};

                \node[block, below left=of 3] (4) {4};
                \node[block, below right=of 3] (5) {5};

                \node[block, below=of 5] (8) {8};
                \node[block, below left=of 5] (6) {6};
                \node[block, below right=of 5] (7) {7};

                \node[block, below left=of 8] (9) {9};
                \node[block, below right=of 8] (10) {10};

                \path[auto=false]
                  (S) edge node[above left=-3pt] {T} (E)
                      edge node[near end, above left=-4pt] {T} (1)
                      edge node[ left=0pt] {T} (2)
                      edge node[near end, above right=-4pt] {T} (3)
                      edge node[above right=-3pt] {T} (12);
                \path (3) edge node[swap] {T} (4)
                          edge[bend left] node {T} (5);
                \path (5) edge  node[swap] {T} (6)
                          edge[bend left] node {F} (3)
                          edge node {F} (8)
                          edge node {F} (7);
                \path (8) edge node[swap] {T} (9)
                          edge node {F} (10);
              \end{tikzpicture}
            \end{center}
            Control dependence graph, without regions.
          \end{minipage}
        \end{minipage}

        \newpage
      \item Below, regions (indicated as trapezoids) are added to the control
        dependence graph. This was performed in one pass through
        the control dependence graph: for each node $n$ in the graph,
        one region was created to group the nodes dependent on $n$
        evaluating to |true|, if any, and another region to group the
        nodes dependent on $n$ evaluating to |false|, if any. Each
        region was made the parent of its corresponding node group,
        and the region in turn was made a child of $n$.

        \begin{minipage}{\linewidth}
          \centering
          \begin{minipage}{.7\linewidth}
            \begin{center}
              \begin{tikzpicture}
                [
                  node distance=.7 and .6,
                  region/.style={trapezium, draw}
                ]
                \node[terminal] (S) {start};

                \node[region, below=of S] (R1) {$R_1$};
                \node[block, below=of R1] (2) {2};
                \node[block, left=of 2]  (1) {1};
                \node[terminal, left=of 1]  (E) {entry};
                \node[block, right=of 2]  (3) {3};
                \node[block, right=of 3]  (12) {12};

                \node[region, below=of 3] (R2) {$R_2$};
                \node[block, below left=of R2] (4) {4};
                \node[block, below right=of R2] (5) {5};

                \node[region, below left=of 5] (R3) {$R_3$};
                \node[region, below right=of 5] (R4) {$R_4$};
                \node[block, below=of R3] (6) {6};
                \node[block, below left=of R4] (7) {7};
                \node[block, below right=of R4] (8) {8};

                \node[region, below left=of 8] (R5) {$R_5$};
                \node[region, below right=of 8] (R6) {$R_6$};
                \node[block, below=of R5] (9) {9};
                \node[block, below=of R6] (10) {10};

                \path (S) edge node[swap] {T} (R1);
                \path (R1)
                      edge (E)
                      edge (1)
                      edge (2)
                      edge (3)
                      edge (12);

                \path (3) edge node[swap] {T} (R2);
                \path (R2) edge (4) edge (5);

                \path (5) edge node[swap] {T} (R3);
                \path (5) edge node       {F} (R4);
                \path (R3) edge (6);
                \path (R4) edge[bend right] (3) edge (7) edge (8);


                \path (8) edge node[swap] {T} (R5);
                \path (8) edge node       {F} (R6);
                \path (R5) edge (9);
                \path (R6) edge (10);
              \end{tikzpicture}
            \end{center}
            Control dependence graph, with regions added.
          \end{minipage}
        \end{minipage}

    \end{enumerate}

  \item
    \newcommand\bottom{\ensuremath{\perp}}
    \newcommand\union{\bigcup}
    \newcommand\Kill{\texttt{KILL}}
    \newcommand\Gen{\texttt{GEN}}
    \begin{enumerate}
      \item Our algorithm assumes that each block contains a single
        statement, so that we can be sure each block contains a single
        definition. Let $S$ be an array that contains, for each block
        (statement) in the old program, the tuple $(\ell, d_\ell,
        k_\ell)$, where
        \begin{itemize}
          \item[$\ell$] is the statement number (and block number),
          \item[$d_\ell$]is the variable that statement $\ell$ defines
            (possibly none, represented by the special symbol
            \bottom), and
          \item[$k_\ell$] is the \Kill\ set for the block at statement
            $\ell$ (possibly empty).
        \end{itemize}
        For our algorithm to work, we need to store the latest value
        of $S$ between iterations of the program.
      \item We assume data about the changes to the program are given
        as a list $C$ containing $(\ell, d'_\ell)$ tuples, where
        $\ell$ is the line number that changed, and $d'_\ell$ is the
        variable that $\ell$ defines in the new program. The value of
        $d'_\ell$ can be \bottom\ if there is no definition at $\ell$,
        and it can also be the same as $d_\ell$, the variable defined
        by $\ell$ in the old program, if the line was altered but
        still defines the same variable. The following algorithm will
        update the relevant \Kill\ sets in $S$.

        \begin{codebox}
          \Procname{
            \proc{Update-Kill}
            $(
              S=[(\ell_i, d_{\ell_i}, k_{\ell_i}), \dots],
              C=[(\ell_j, d'_{\ell_j}), \dots]
            )$
          }
          \li \For each $(\ell_i, d'_{\ell_i})$ in $C$
          \li   \Do
                  Find $(\ell_i, d_{\ell_i} k_{\ell_i})$ in $S$
          \li     \If $d_{\ell_i} \ne d'_{\ell_i}$                                         \label{li:check-def}
          \li       \Then
                      \For each $(\ell_j, d_{\ell_j}, k_{\ell_j})$ in $S$
          \li           \Do
                          \If $d_{\ell_j} = d'_{\ell_i} \ne \bottom$, $\ell_j \ne \ell_i$  \label{li:add-kill-start}
          \li               \Then
                              Let $k_{\ell_j} \gets k_{\ell_j} \union \ell_i$
          \li                 Let $k_{\ell_i} \gets k_{\ell_i} \union \ell_j$              \label{li:add-kill-end}
          \li             \ElseIf $\ell_i \in k_{\ell_j}$                                      \label{li:rm-kill-start}
          \li               \Then
                              Let $k_{\ell_j} \gets k_{\ell_j} - \ell_i$
          \li                 Let $k_{\ell_i} \gets k_{\ell_i} - \ell_j$                   \label{li:rm-kill-end}
                          \End
                      \End
          \li         Let $d_{\ell_i} \gets d'_{\ell_i}$                                   \label{li:update-def}
                  \End
              \End
        \end{codebox}

        This procedure iterates through every line of change $\ell_i$
        in $C$.  In line~\ref{li:check-def}, it first checks to see if
        the definition at the line has really been changed, as these
        changes are the only ones that will affect reaching
        definitions. It then iterates through $S$. If a different
        block defines the same variable as $\ell_i$
        (and they actually give a definition; that is, neither are
        \bottom), then both \Kill\ sets need to be updated to include
        each other
        (lines~\ref{li:add-kill-start}--\ref{li:add-kill-end}).
        Alternatively, if a block is found with $\ell_i$ in its \Kill\
        set, the both blocks need to be removed from each other's
        \Kill\ set
        (lines~\ref{li:rm-kill-start}--\ref{li:rm-kill-end})---we know
        that the definitions are different from the check in
        line~\ref{li:check-def}, and statements that define different
        variables can't kill each other. Finally, the definition in
        $S$ is updated to the new definition from $C$
        (line~\ref{li:update-def}).

        After running this procedure, $S$ will contain the \Kill\ sets
        for each block, and the \Gen\ set for block $\ell$ will just
        be $\s{\ell}$ if $d_{\ell} \ne \bottom$, else $\emptyset$.
        So, since we have the \Gen\ and \Kill\ sets for every block in
        the new program, we can compute the reaching definitions for
        the new version of the program using the usual algorithm.

      \item \todo{work through the algo. by hand for the test case}

    \end{enumerate}
\end{enumerate}

\newlength\fatcolumn

\begin{table}[htbp]
  \caption
  {
    Reaching definitions, problem 1(a). Definitions are denoted
    by their block number (source code line number); definitions made
    in the entry block (that is, program input parameters) are
    denoted by $E$. We define $D$ to be the set of all definitions.
    The notation \s{$i$--$k$} means \s{$i, i+1, \dots, k$}. Changes
    from iteration 1 to iteration 2 are highlighted in red.
  }
  \label{reaching-defs}
  \makebox[\textwidth]{% centers large table on page
  \small
  \setlength\fatcolumn{16.5ex}
  %               Block Gen Kill I:In Out 1:In          Out           2:In          Out
  \begin{tabular}{c     c   c    c    c   M{\fatcolumn} M{\fatcolumn} M{\fatcolumn} M{\fatcolumn}}
    \toprule
    \multirow{2}{*}{Block}
    & \multirow{2}{*}{Gen}
    & \multirow{2}{*}{Kill}
    & \multicolumn{2}{c}{Init}
    & \multicolumn{2}{c}{Iteration 1}
    & \multicolumn{2}{c}{Iteration 2}
    \\
    \cmidrule(lr){4-5}
    \cmidrule(lr){6-7}
    \cmidrule(lr){8-9}
    & &
    & In & Out
    & In & Out
    & In & Out
    \\
    \midrule
    entry  & \s{$E$}     & $\emptyset$ & $\emptyset$ & \s{$E$}     & $\emptyset$                                      & \s{$E$}                                          & $\emptyset$                                           & \s{$E$}                                               \newrow
    1      & \s{1}       & \s{16}      & $\emptyset$ & \s{1}       & \s{$E$}                                          & \s{$E$, 1}                                       & \s{$E$}                                               & \s{$E$, 1}                                            \newrow
    2      & \s{2}       & $\emptyset$ & $\emptyset$ & \s{2}       & \s{$E$, 1}                                       & \s{$E$, 1, 2}                                    & \s{$E$, 1}                                            & \s{$E$, 1, 2}                                         \newrow
    3      & \s{3}       & $\emptyset$ & $\emptyset$ & \s{3}       & \s{$E$, 1, 2}                                    & \s{$E$, 1--3}                                    & \s{$E$, 1, 2}                                         & \s{$E$, 1--3}                                         \newrow
    4      & $\emptyset$ & $\emptyset$ & $\emptyset$ & $\emptyset$ & \s{$E$, 1--3}                                    & \s{$E$, 1--3}                                    & \s{$E$, 1--3}                                         & \s{$E$, 1--3}                                         \newrow
    5      & $\emptyset$ & $\emptyset$ & $\emptyset$ & $\emptyset$ & \s{$E$, 1--3}                                    & \s{$E$, 1--3}                                    & \s{$E$, 1--3}                                         & \s{$E$, 1--3}                                         \newrow
    6      & $\emptyset$ & $\emptyset$ & $\emptyset$ & $\emptyset$ & \s{$E$, 1--3}                                    & \s{$E$, 1--3}                                    & \s{$E$, 1--3}                                         & \s{$E$, 1--3}                                         \newrow
    8      & $\emptyset$ & $\emptyset$ & $\emptyset$ & $\emptyset$ & \s{$E$, 1--3}                                    & \s{$E$, 1--3}                                    & \s{$E$, 1--3}                                         & \s{$E$, 1--3}                                         \newrow
    9      & \s{9}       & \s{12, 20}  & $\emptyset$ & \s{9}       & \s{$E$, 1--3}                                    & \s{$E$, 1--3, 9}                                 & \s{$E$, 1--3}                                         & \s{$E$, 1--3, 9}                                      \newrow
    10     & \s{10}      & \s{13, 18}  & $\emptyset$ & \s{10}      & \s{$E$, 1--3, 9}                                 & \s{$E$, 1--3, 9, 10}                             & \s{$E$, 1--3, 9}                                      & \s{$E$, 1--3, 9, 10}                                  \newrow
    12     & \s{12}      & \s{9, 20}   & $\emptyset$ & \s{12}      & \s{$E$, 1--3}                                    & \s{$E$, 1--3, 12}                                & \s{$E$, 1--3}                                         & \s{$E$, 1--3, 12}                                     \newrow
    13     & \s{13}      & \s{10, 18}  & $\emptyset$ & \s{13}      & \s{$E$, 1--3, 12}                                & \s{$E$, 1--3, 12, 13}                            & \s{$E$, 1--3, 12}                                     & \s{$E$, 1--3, 12, 13}                                 \newrow
    15     & $\emptyset$ & $\emptyset$ & $\emptyset$ & $\emptyset$ & \s{$E$, 1--3, 9, 10,\newline 12, 13, 18, 20}     & \s{$E$, 1--3, 9, 10,\newline 12, 13, 18, 20}     & \s{$E$, 1--3, 9, 10,\newline 12, 13, \hi{16}, 18, 20} & \s{$E$, 1--3, 9, 10,\newline 12, 13, \hi{16}, 18, 20} \newrow
    16     & \s{16}      & \s{1}       & $\emptyset$ & \s{16}      & \s{$E$, 1--3, 9, 10,\newline 12, 13, 18, 20}     & \s{$E$, 2, 3, 9, 10,\newline 12, 13, 16, 18, 20} & \s{$E$, 1--3, 9, 10,\newline 12, 13, \hi{16}, 18, 20} & \s{$E$, 2, 3, 9, 10,\newline 12, 13, 16, 18, 20}      \newrow
    17     & $\emptyset$ & $\emptyset$ & $\emptyset$ & $\emptyset$ & \s{$E$, 2, 3, 9, 10,\newline 12, 13, 16, 18, 20} & \s{$E$, 2, 3, 9, 10,\newline 12, 13, 16, 18, 20} & \s{$E$, 2, 3, 9, 10,\newline 12, 13, 16, 18, 20}      & \s{$E$, 2, 3, 9, 10,\newline 12, 13, 16, 18, 20}      \newrow
    18     & \s{18}      & \s{10, 13}  & $\emptyset$ & \s{18}      & \s{$E$, 2, 3, 9, 10,\newline 12, 13, 16, 18, 20} & \s{$E$, 2, 3, 9,\newline 12, 16, 18, 20}         & \s{$E$, 2, 3, 9, 10,\newline 12, 13, 16, 18, 20}      & \s{$E$, 2, 3, 9,\newline 12, 16, 18, 20}              \newrow
    20     & \s{20}      & \s{9, 12}   & $\emptyset$ & \s{20}      & \s{$E$, 2, 3, 9, 10,\newline 12, 13, 16, 18, 20} & \s{$E$, 2, 3, 10,\newline 13, 16, 18, 20}        & \s{$E$, 2, 3, 9, 10,\newline 12, 13, 16, 18, 20}      & \s{$E$, 2, 3, 10,\newline 13, 16, 18, 20}             \newrow
    23     & $\emptyset$ & $\emptyset$ & $\emptyset$ & $\emptyset$ & \s{$E$, 1--3, 9, 10,\newline 12, 13, 18, 20}     & \s{$E$, 1--3, 9, 10,\newline 12, 13, 18, 20}     & \s{$E$, 1--3, 9, 10,\newline 12, 13, \hi{16}, 18, 20} & \s{$E$, 1--3, 9, 10,\newline 12, 13, \hi{16}, 18, 20} \newrow
    exit   & $\emptyset$ & $D$         & $\emptyset$ & $\emptyset$ & \s{$E$, 1--3, 9, 10,\newline 12, 13, 18, 20}     & $\emptyset$                                      & \s{$E$, 1--3, 9, 10,\newline 12, 13, \hi{16}, 18, 20} & $\emptyset$                                           \newrow
    \bottomrule
  \end{tabular}
  } % end makebox
\end{table}

\newgeometry{vmargin=1in,hmargin=.5in}
\begin{landscape}
  \begin{table}[htbp]
    \centering
    \begin{minipage}{\textwidth}
      \caption
      {
        Reachable uses, problem 1(b). If a source code line contains
        only one use, the use is denoted by the line number.
        Otherwise, the notation $\ell.i$ means the $i$th use in line
        $\ell$, and \s{$\ell.i$--$k$} means \s{$\ell.i, \ell.(i+1),
        \dots \ell.k$}. Changes from iteration 1 to iteration 2 are
        highlighted in red.
      }
      \label{reachable-uses}
      \makebox[\textwidth]{% centers large table on page
      \small
      \setlength\fatcolumn{19ex}
      %               Block Gen Kill          I:In Out           1:In          Out           2:In          Out
      \begin{tabular}{c     c   M{\fatcolumn} c    M{\fatcolumn} M{\fatcolumn} M{\fatcolumn} M{\fatcolumn} M{\fatcolumn}}
        \toprule
        \multirow{2}{*}{Block}
        & \multirow{2}{*}{Gen}
        & \multirow{2}{*}{Kill}
        & \multicolumn{2}{c}{Init}
        & \multicolumn{2}{c}{Iteration 1}
        & \multicolumn{2}{c}{Iteration 2}
        \\
        \cmidrule(lr){4-5}
        \cmidrule(lr){6-7}
        \cmidrule(lr){8-9}
        & &
        & In & Out
        & In & Out
        & In & Out
        \\
        \midrule
        entry & $\emptyset$ & \s{4, 8, 9, 13,\newline 17.3, 17.6} & $\emptyset$ & \s{4, 8, 9, 13,\newline 17.3, 17.6} & \s{4, 8, 9, 13,\newline 17.3, 17.6}                  & $\emptyset$                                          & \s{4, 8, 9, 13,\newline 17.3, 17.6}                               & $\emptyset$                                                          \newrow
        1     & $\emptyset$ & \s{17.1--2,\newline 18, 20, 23}     & $\emptyset$ & $\emptyset$                         & \s{4, 8, 9, 13,\newline 17.3, 17.6, 23}              & \s{4, 8, 9, 13,\newline 17.3, 17.6}                  & \s{4, 8, 9, 13,\newline 17.3, 17.6, 23}                           & \s{4, 8, 9, 13,\newline 17.3, 17.6}                                  \newrow
        2     & $\emptyset$ & \s{6}                               & $\emptyset$ & $\emptyset$                         & \s{4, 6, 8, 9, 13,\newline 17.3, 17.6, 23}           & \s{4, 8, 9, 13,\newline 17.3, 17.6, 23}              & \s{4, 6, 8, 9, 13,\newline 17.3, 17.6, 23}                        & \s{4, 8, 9, 13,\newline 17.3, 17.6, 23}                              \newrow
        3     & $\emptyset$ & \s{12, 15.3}                        & $\emptyset$ & $\emptyset$                         & \s{4, 6, 8, 9, 12, 13,\newline 15.3, 17.3, 17.6, 23} & \s{4, 6, 8, 9, 13,\newline 17.3, 17.6, 23}           & \s{4, 6, 8, 9, 12, 13,\newline 15.3, 17.3, 17.6, 23}              & \s{4, 6, 8, 9, 13,\newline 17.3, 17.6, 23}                           \newrow
        4     & \s{4}       & $\emptyset$                         & $\emptyset$ & \s{4}                               & \s{6, 8, 9, 12, 13,\newline 15.3, 17.3, 17.6, 23}    & \s{4, 6, 8, 9, 12, 13,\newline 15.3, 17.3, 17.6, 23} & \s{6, 8, 9, 12, 13,\newline 15.3, 17.3, 17.6, 23}                 & \s{4, 6, 8, 9, 12, 13,\newline 15.3, 17.3, 17.6, 23}                 \newrow
        5     & $\emptyset$ & $\emptyset$                         & $\emptyset$ & $\emptyset$                         & \s{6}                                                & \s{6}                                                & \s{6}                                                             & \s{6}                                                                \newrow
        6     & \s{6}       & $\emptyset$                         & $\emptyset$ & \s{6}                               & $\emptyset$                                          & \s{6}                                                & $\emptyset$                                                       & \s{6}                                                                \newrow
        8     & \s{8}       & $\emptyset$                         & $\emptyset$ & \s{8}                               & \s{9, 12, 13, 15.3,\newline 17.3, 17.6, 23}          & \s{8, 9, 12, 13, 15.3,\newline 17.3, 17.6, 23}       & \s{9, 12, 13, 15.3,\newline 17.3, 17.6, 23}                       & \s{8, 9, 12, 13, 15.3,\newline 17.3, 17.6, 23}                       \newrow
        9     & \s{9}       & \s{15.2, 16.1,\newline 17.4--5}     & $\emptyset$ & \s{9}                               & \s{15.2--3, 16,\newline 17.3--6, 23}                 & \s{9, 15.3, 17.3,\newline 17.6, 23}                  & \s{15.2--3, 16,\newline 17.3--6, 23}                              & \s{9, 15.3, 17.3,\newline 17.6, 23}                                  \newrow
        10    & $\emptyset$ & \s{15.1, 16.2}                      & $\emptyset$ & $\emptyset$                         & \s{15.1--3, 16.1--2,\newline 17.3--6, 23}            & \s{15.2--3, 16.1,\newline 17.3--6, 23}               & \s{15.1--3, 16.1--2,\newline 17.3--6, 23}                         & \s{15.2--3, 16.1,\newline 17.3--6, 23}                               \newrow
        12    & \s{12}      & \s{15.2, 16.1,\newline 17.4--5}     & $\emptyset$ & \s{12}                              & \s{13, 15.2--3,\newline 16, 17.3--6, 23}             & \s{12, 13, 15.3,\newline 17.3, 17.6, 23}             & \s{13, 15.2--3,\newline 16, 17.3--6, 23}                          & \s{12, 13, 15.3,\newline 17.3, 17.6, 23}                             \newrow
        13    & \s{13}      & \s{15.1, 16.2}                      & $\emptyset$ & \s{13}                              & \s{15.1--3, 16.1--2,\newline 17.3--6, 23}            & \s{13, 15.2--3, 16.1,\newline 17.3--6, 23}           & \s{15.1--3, 16.1--2,\newline 17.3--6, 23}                         & \s{13, 15.2--3, 16.1,\newline 17.3--6, 23}                           \newrow
        15    & \s{15.1--3} & $\emptyset$                         & $\emptyset$ & \s{15.1--3}                         & \s{15.1--3, 16.1--2,\newline 17.3--6, 23}            & \s{15.1--3, 16.1--2,\newline 17.3--6, 23}            & \s{15.1--3, 16.1--2,\newline 17.3--6, 23}                         & \s{15.1--3, 16.1--2,\newline 17.3--6, 23}                            \newrow
        16    & \s{16.1--2} & \s{17.1--2,\newline 18, 20, 23}     & $\emptyset$ & \s{16}                              & \s{15.1--3, 17.1--6,\newline 18, 20}                 & \s{15.1--3, 16.1--2,\newline 17.3--6}                & \s{15.1--3, \hi{16.1--2},\newline 17.1--6, 18, 20, \hi{23}}       & \s{15.1--3, 16.1--2,\newline 17.3--6}                                \newrow
        17    & \s{17.1--6} & $\emptyset$                         & $\emptyset$ & \s{17.1--6}                         & \s{15.1--3, 18, 20}                                  & \s{15.1--3,\newline 17.1--6, 18, 20}                 & \s{15.1--3, \hi{16.1--2},\newline \hi{17.3--6},  18, 20, \hi{23}} & \s{15.1--3, \hi{16.1--2},\newline 17.1--6, 18, 20, \hi{23}}          \newrow
        18    & \s{18}      & \s{15.1, 16.2}                      & $\emptyset$ & \s{18}                              & \s{15.1--3}                                          & \s{15.1, 15.3, 18}                                   & \s{15.1--3, \hi{16.1--2},\newline \hi{17.3--6}, \hi{23}}          & \s{15.2--3, \hi{16.1},\newline \hi{17.3--6}, 18, \hi{23}}            \newrow
        20    & \s{20}      & \s{15.2, 16.1,\newline 17.4--5}     & $\emptyset$ & \s{20}                              & \s{15.1--3}                                          & \s{15.1, 15.3, 20}                                   & \s{15.1--3, \hi{16.1--2},\newline \hi{17.3--6}, \hi{23}}          & \s{15.1, 15.3, \hi{16.2},\newline \hi{17.3}, \hi{17.6}, 20, \hi{23}} \newrow
        23    & \s{23}      & $\emptyset$                         & $\emptyset$ & \s{23}                              & $\emptyset$                                          & \s{23}                                               & $\emptyset$                                                       & \s{23}                                                               \newrow
        \bottomrule
      \end{tabular}
      } % end makebox
    \end{minipage}
  \end{table}
\end{landscape}
\restoregeometry

\begin{table}[htbp]
  \centering
  \begin{minipage}[t]{.7\linewidth}
    \centering
  \caption
  {
    Definition-use pairs, problem 1(c). If a source code line contains
    only one use, the use is denoted by the line number. Otherwise,
    the notation $\ell.i$ means the $i$th use in line $\ell$.
    Definitions in the entry block (that is, program parameters) are
    denoted by $E$.
  }
  \label{def-use-pairs}
  \small
  \begin{tabular}{c>{\ttfamily}cc}
    \toprule
    Use
    & \multicolumn{1}{c}{Variable}
    & Definitions
    \\
    \midrule
    4    & x      & \s{$E$}        \\
    6    & errval & \s{2}          \\
    8    & x      & \s{$E$}        \\
    9    & x      & \s{$E$}        \\
    12   & eps    & \s{3}          \\
    13   & x      & \s{$E$}        \\
    15.1 & x2     & \s{10, 13, 18} \\
    15.2 & x1     & \s{9, 12, 20}  \\
    15.3 & eps    & \s{3}          \\
    16.1 & x1     & \s{9, 12, 20}  \\
    16.2 & x2     & \s{10, 13, 18} \\
    17.1 & x3     & \s{16}         \\
    17.2 & x3     & \s{16}         \\
    17.3 & x      & \s{$E$}        \\
    17.4 & x1     & \s{9, 12, 20}  \\
    17.5 & x1     & \s{9, 12, 20}  \\
    17.6 & x      & \s{$E$}        \\
    18   & x3     & \s{16}         \\
    20   & x3     & \s{16}         \\
    23   & x3     & \s{1, 16}      \\
    \bottomrule
  \end{tabular}
  \end{minipage}
\end{table}



\end{document}
