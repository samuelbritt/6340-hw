\documentclass{article}

\usepackage[T1]{fontenc}
\usepackage[osf]{mathpazo}
\usepackage{microtype}

\usepackage{amsmath}
\usepackage{amssymb}
\usepackage{amsfonts}
\usepackage{amsthm}
\usepackage{mathtools}
\usepackage{bm}
\usepackage{xcolor}
\usepackage{graphicx}
\usepackage[pdftex]{lscape}
\usepackage{lipsum}
\usepackage{tikz}
\usepackage{enumitem}
\usepackage[alsoload=binary]{siunitx}
\usepackage{booktabs}
\usepackage{multirow}
\usepackage{tabularx}
\usepackage{cancel}
\usepackage{fancyvrb}
\usepackage[section]{placeins}
\usepackage{flafter}
\usepackage{pdfpages}
\usepackage[hmargin=1.5in,vmargin=1in]{geometry}
\usepackage{url}
\usepackage{hyperref}

% algorithms
\let\oldgets\gets
\usepackage{clrscode3e}
\let\gets\oldgets

% no section numbers
\setcounter{secnumdepth}{-2}

% a bit more compact, and fixes spacing issues
\let\originalleft\left
\let\originalright\right
\renewcommand{\l}{\mathopen{}\mathclose\bgroup\originalleft}
\renewcommand{\r}{\aftergroup\egroup\originalright}

% commands
\renewcommand\vec[1]{\bm{#1}}
\renewcommand\epsilon{\varepsilon}
\newcommand\bit{\ensuremath{\set{0,1}}}
\newcommand\abs[1]{\l\vert #1 \r\vert}
\newcommand\set[1]{\ensuremath{\l\{#1\r\}}}

% red todos
\usepackage{marginnote}
\usepackage{xifthen}% provides \isempty test
\renewcommand\marginfont{\color{red}}

% Arg 1: vertical shift, Arg 2: content
\newcommand\Todo[2][]{{\color{red}$^\dagger$}\marginnote{$^\dagger$#2}[#1]}

% wrapper for \Todo: can leave content and/or shift empty
\newcommand{\todo}[2][0pt]
  {%
    \ifthenelse{ \isempty{#2} }%
      {%
        \Todo[#1]{\textsc{todo}}%
      }%
      {%
        \Todo[#1]{#2}%
      }%
  }

% write in code
\DefineShortVerb{\|}

\usetikzlibrary
{
  positioning,
  calc,
  intersections,
  arrows,
  shapes.misc,
  shapes.geometric,
}

% redefine `\to` to use tikz arrows
\renewcommand\to{%
  \mathbin{\tikz[baseline=-.7ex] \draw[->] (0,0) -- +(.4,0);}%
}

\tikzset
  {
    ->,
    >=latex',
    shorten >=.5pt,
    node distance=.5cm and .7cm,
    auto,
    fork/.style= {node distance=.5cm and #1},
    wide fork/.style={fork=1.2cm},
    block/.style=
      {
        draw,
        inner sep=.3em,
        rounded rectangle,
        minimum size=1.7em,
      },
    terminal/.style=
      {
        block,
        rectangle,
        text height=1.5ex,
        text depth=.25ex,
      },
    region/.style=
      {
        draw,
        trapezium,
      },
    inline center/.style={baseline=-.5ex},
    retreating/.style={dashed},
    cross/.style={dashdotted},
    forward/.style={dotted},
  }

% table customizaton
\newcommand\newrow{\\\addlinespace}

% center columns
\newcolumntype{M}[1]{>{\centering\arraybackslash}m{#1}}

% set
\newcommand\s[1]{\{#1\}}

% highlight
\newcommand\hi[1]{\textcolor{red}{#1}}


% data analysis
\newcommand\bottom{\ensuremath{\perp}}
\newcommand\union{\bigcup}
\newcommand\Kill{\texttt{KILL}}
\newcommand\Gen{\texttt{GEN}}
\newcommand\varfamily{\ttfamily}  % use in tables
\newcommand\var[1]{\text{\varfamily #1}}
\DeclareMathOperator\pdom{pdom}

\begin{document}
\thispagestyle{empty}

\includepdf
  [
    pages=1,
    pagecommand=
      {
        \begin{tikzpicture}[remember picture, overlay]
          \node[text width=6.3in, text height=2.6in]
          at (current page.north east)
          {\large
            Sam Britt, Shriram Swaminathan,\\[1.5em]
            and Sivaramachandran Ganesan
          };
        \end{tikzpicture}
      }
  ]
  {Assignment4ProblemSet3}

\clearpage
\pagenumbering{arabic}

\noindent
Sam Britt                \hfill CS 6340      \\
Shriram Swaminathan      \hfill Assignment 4 \\
Sivaramachandran Ganesan \hfill Jan. 30, 2013

\begin{enumerate}
  \item We start with the control dependence graph, augmented with a
    ``start'' node (below, left). From this, we create the
    postdominator tree (below, right).

    \hfill
    \begin{minipage}[b]{.4\linewidth}
      \begin{center}
        \begin{tikzpicture}
          \node[terminal]                 (E)  {entry};
          \node[block,below=of E]         (1)  {1};
          \node[block,below=of 1]         (2)  {2};
          \node[block,below=of 2]         (3)  {3};
          \node[block,below=of 3]         (4)  {4};
          \node[block,below right=of 4]   (8)  {8};
          \node[block,below  left=of 4]   (5)  {5};
          \node[block,below  left=of 5]   (6)  {6};
          \node[block,below  left=of 8]   (12) {12};
          \node[block,below=of 12]        (13) {13};
          \node[block,below right=of 8]   (9)  {9};
          \node[block,below=of 9]         (10) {10};
          \node[block,below right=of 13]  (15) {15};
          \node[block,below  left=of 15]  (23) {23};
          \node[terminal,below=of 23]     (X)  {exit};
          \node[block,below right=of 15]  (16) {16};
          \node[block,below=of 16]        (17) {17};
          \node[block,below  left=of 17]  (20) {20};
          \node[block,below right=of 17]  (18) {18};
          \node[block,below right=of 20]  (J)  {J};
          \path (E) -- node (mid) {} (X);
          \node[terminal,left=1.6 of mid]   (S)  {start};

          \path (E) edge (1);
          \path (1) edge (2);
          \path (2) edge (3);
          \path (3) edge (4);
          \path (4) edge node[swap] {T} (5)
                    edge node {F} (8);
          \path (5) edge (6);
          \path (8) edge node[swap] {F} (12)
                    edge node {T} (9);
          \path (9) edge (10);
          \path (10) edge (15);
          \path (12) edge (13);
          \path (13) edge (15);
          \path (15) edge node[swap] {F} (23)
                     edge node {T} (16);
          \path (23) edge (X);
          \path (16) edge (17);
          \path (17) edge node[swap] {F} (20)
                     edge node {T} (18);
          \path (18) edge (J);
          \path (20) edge (J);

          \draw (S) |- node[left,near start] {T} (E);
          \draw (S) |- node[left,near start] {F} (X.190);

          \draw (6) |- (X.170);
          \draw (J) -- ++(1.3,0) |- (15);
        \end{tikzpicture}
      \end{center}
      Augmented CFG.
    \end{minipage}
    \hfill
    \begin{minipage}[b]{.55\linewidth}
      \begin{center}
        \begin{tikzpicture}[on grid,node distance=1 and 1]
          \node[block] (4)  {4};
          \node[block, right=of 4] (6) {6};
          \node[block, right=of 6] (23) {23};
          \node[terminal, right=of 23] (S)  {start};
          \node[terminal] at ($(6)!.5!(23) + (0,1)$) (X)  {exit};

          \node[block, below left=of 4] (3) {3};
          \node[block, below left=1 and .5 of 3] (2) {2};
          \node[block, below=of 2] (1) {1};
          \node[terminal, below=of 1] (E) {entry};

          \node[block, below left=of 6] (5) {5};

          \node[block, below= of 23] (15) {15};
          \node[block, below right=1 and .5 of 15] (17) {17};
          \node[block, right=of 17] (18) {18};
          \node[block, right=of 18] (20) {20};
          \node[block, below left=1 and .5 of 15] (13) {13};
          \node[block, left=of 13] (10) {10};
          \node[block, left=of 10] (8) {8};

          \node[block, below=of 10] (9) {9};
          \node[block, below=of 13] (12) {12};
          \node[block, below=of 17] (16) {16};

          \draw (X) -- (4);
          \draw (4) -- (3);
          \draw (3) -- (2);
          \draw (2) -- (1);
          \draw (1) -- (E);

          \draw (X) -- (6);
          \draw (6) -- (5);

          \draw (X) -- (23);
          \draw (23) -- (15);
          \draw (15) -- (8);
          \draw (15) -- (10);
          \draw (15) -- (13);
          \draw (15) -- (17);
          \draw (15) -- (18);
          \draw (15) -- (20);
          \draw (10) -- (9);
          \draw (13) -- (12);
          \draw (17) -- (16);

          \draw (X) -- (S);
        \end{tikzpicture}
      \end{center}
      Corresponding postdominator tree.
    \end{minipage}
    \hfill~

    We then find the set
    $S = \set{ (A, B) \colon (A, B) \in G, \overline{B \pdom A} }$, where
    $G$ is the above augmented CFG. We find that
    \begin{equation*}
      S = \set{\text{(start, entry), (4, 5), (4, 8), (8, 9), (8, 12),
          (15, 16), (17, 18), (17, 20)}}.
    \end{equation*}

    For each edge $(A,B) \in S$, we find $L$ to be the common ancestor
    of $A$ and $B$. Finally, the nodes that are control-dependent on
    $A$ are those the path from $L$ to $B$ on the postdominator tree,
    including $B$, and including $L$ only if $L=A$. These results are
    summarized in the following table. Also noted is ``condition,''
    the label on the edge $(A, B)$.
    \begin{center}
      \begin{tabular}{cccl}
        \toprule
        $(A,B) \in S$ & Condition (T/F) & $L$ & Nodes dependent on
        $A$ \\
        \midrule
        (start, entry) & T & exit & \s{entry, 1, 2, 3, 4} \\
        (4, 5)         & T & exit & \s{5, 6} \\
        (4, 8)         & F & exit & \s{8, 15, 23} \\
        (8, 9)         & T & 15   & \s{9, 10} \\
        (8, 12)        & F & 15   & \s{12, 13} \\
        (15, 16)       & T & 15   & \s{15, 16, 17} \\
        (17, 18)       & T & 15   & \s{18} \\
        (17, 20)       & F & 15   & \s{20} \\
        \bottomrule
      \end{tabular}
    \end{center}

    The control dependence graph (below) is constructed directly from the
    above table.
    \begin{center}
      \begin{minipage}[b]{.8\linewidth}
        \begin{center}
          \begin{tikzpicture}[node distance=1.4 and .8]
            \node[terminal] (S) {start};
            \node[block, below=of S] (2) {2};
            \node[block, left=of 2] (1) {1};
            \node[terminal, left=of 1] (E) {entry};
            \node[block, right=of 2] (3) {3};
            \node[block, right=of 3] (4) {4};

            \node[block, below=of 4] (8) {8};
            \node[block, left=of 8] (6) {6};
            \node[block, left=of 6] (5) {5};
            \node[block, right=of 8] (15) {15};
            \node[block, right=of 15] (23) {23};

            \node[block, below=of 8] (13) {13};
            \node[block, left=of 13] (12) {12};
            \node[block, left=of 12] (10) {10};
            \node[block, left=of 10] (9) {9};

            \node[block, below=of 15] (16) {16};
            \node[block, right=of 16] (17) {17};

            \node[block, below left=of 17] (18) {18};
            \node[block, below right=of 17] (20) {20};

            \path (S)
            edge node[swap,very near end] {T} (E)
            edge node[near end, left=.1] {T} (1)
            edge node[near end] {T} (2)
            edge node[near end, right=.1] {T} (3)
            edge node[very near end] {T} (4);

            \path (4)
            edge node[swap,very near end] {T} (5)
            edge node[near end, left=.1] {T} (6)
            edge node[near end] {F} (8)
            edge node[near end, right=.1] {F} (15)
            edge node[very near end] {F} (23);

            \path (8)
            edge node[swap,very near end] {T} (9)
            edge node[swap,near end, left=.1] {T} (10)
            edge node[near end,left=.1] {F} (12)
            edge node[near end] {F} (13);

            \path (15)
            edge node[near end] {T} (16)
            edge node[near end,right=.1] {T} (17)
            edge[loop left] node[below=.1] {T} (15);

            \path (17)
            edge node[near end, swap] {T} (18)
            edge node[near end] {F} (20);
          \end{tikzpicture}
        \end{center}
        Control dependence graph constructed using the FOW method.
      \end{minipage}
    \end{center}

    \newpage
  \item We start by augmenting the CFG with an edge from entry to
    exit, and then reverse the entire graph (below, left). From that
    we create the dominator tree (below, right).

    \hfill
    \begin{minipage}[b]{.5\linewidth}
      \begin{center}
        \begin{tikzpicture}[on grid, node distance=1 and 1]
          \node[terminal] (X) {exit};
          \node[block, below left=of X] (6) {6};
          \node[block, below right=of X] (23) {23};

          \node[block, below left=of 6] (5) {5};
          \node[block, below=of 5] (4) {4};
          \node[block, below=of 4] (3) {3};
          \node[block, below=of 3] (2) {2};
          \node[block, below=of 2] (1) {1};
          \node[terminal, below=of 1] (E) {entry};

          \node[block, below=of 23] (15) {15};
          \node[block, below left=1 and .5 of 15] (13) {13};
          \node[block, left=of 13] (10) {10};
          \node[block, right=of 13] (18) {18};
          \node[block, right=of 18] (20) {20};

          \node[block, below=of 10] (9) {9};
          \node[block, below=of 13] (12) {12};
          \node[block, below right=1 and .5 of 9] (8) {8};

          \node[block, below right=1 and .5 of 18] (17) {17};
          \node[block, below=of 17] (16) {16};

          \draw (X) -- (6);
          \draw (6) -- (5);
          \draw (5) -- node {T} (4);
          \draw (4) -- (3);
          \draw (3) -- (2);
          \draw (2) -- (1);
          \draw (1) -- (E);

          \draw (X) -- (23);
          \draw (23) -- node {F} (15);

          \draw (15) -- (10);
          \draw (15) -- (13);
          \draw (15) -- (18);
          \draw (15) -- (20);

          \draw (10) -- (9);
          \draw (9) -- node[left=-.03] {T} (8);
          \draw (13) -- (12);
          \draw (12) -- node[right=-.03] {F} (8);
          \path (8) edge[out=180,in=-60] node[below left=-.1,very near start] {F} (4);

          \draw (18) -- node[swap] {T} (17);
          \draw (20) -- node {F} (17);
          \draw (17) -- (16);

          \draw (16) -- ++(1,0) |- node[swap, near start] {T} (15);

          \draw (X) -- (-3,0) |- (E);

        \end{tikzpicture}
      \end{center}
      Reverse CFG.
    \end{minipage}
    \hfill
    \begin{minipage}[b]{.45\linewidth}
      \begin{center}
        \begin{tikzpicture}[on grid,node distance=1 and 1]
          \node[terminal] (X) {exit};
          \node[block, below left=1 and .5 of X] (6) {6};
          \node[block, left=of 6] (4)  {4};
          \node[block, right=of 6] (23) {23};
          \node[terminal, right=of 23] (E) {entry};

          \node[block, below left=of 4] (3) {3};
          \node[block, below left=1 and .5 of 3] (2) {2};
          \node[block, below=of 2] (1) {1};

          \node[block, below left=of 6] (5) {5};

          \node[block, below= of 23] (15) {15};
          \node[block, below right=1 and .5 of 15] (17) {17};
          \node[block, right=of 17] (18) {18};
          \node[block, right=of 18] (20) {20};
          \node[block, below left=1 and .5 of 15] (13) {13};
          \node[block, left=of 13] (10) {10};
          \node[block, left=of 10] (8) {8};
          \node[block, below=of 10] (9) {9};
          \node[block, below=of 13] (12) {12};
          \node[block, below=of 17] (16) {16};

          \draw (X) -- (4);
          \draw (4) -- (3);
          \draw (3) -- (2);
          \draw (2) -- (1);

          \draw (X) -- (6);
          \draw (6) -- (5);

          \draw (X) -- (23);
          \draw (23) -- (15);
          \draw (15) -- (8);
          \draw (15) -- (10);
          \draw (15) -- (13);
          \draw (15) -- (17);
          \draw (15) -- (18);
          \draw (15) -- (20);
          \draw (10) -- (9);
          \draw (13) -- (12);
          \draw (17) -- (16);

          \draw (X) -- (E);
        \end{tikzpicture}
      \end{center}
      Corresponding dominator tree.
    \end{minipage}
    \hfill~

    For each node $n$, we traverse down the reverse CFG (RCFG) until
    we find a node that $n$ does not dominate; that is, a node that is
    a child of $n$ in the RCFG but not a child of $n$ in the dominator
    tree. This set of nodes is the dominance frontier of $n$. These
    are tabulated below.\todo{mention `condition' if including}

    \begin{center}
      \begin{minipage}[c]{.4\linewidth}
        \begin{tabular}{ccc}
          \toprule
          \multirow{2}{*}{Block}
          & Dominance
          & \multirow{2}{*}{Condition\todo{does ``condition'' even make
      sense?}} \\
          & frontier & \\
          \midrule
          entry & $\emptyset$& T  \\
          1     & \s{entry}  & T  \\
          2     & \s{entry}  & T  \\
          3     & \s{entry}  & T  \\
          4     & \s{entry}  & T  \\
          5     & \s{4}      & T  \\
          6     & \s{4}      & T  \\
          8     & \s{4}      & F  \\
          9     & \s{8}      & T  \\
          10    & \s{8}      & T  \\
          12    & \s{8}      & F  \\
          13    & \s{8}      & F  \\
          15\todo{does 15 dominate 15?}    & \s{4}      & F  \\
          16    & \s{15}     & T   \\
          17    & \s{15}     & T   \\
          18    & \s{17}     & T   \\
          20    & \s{17}     & F   \\
          23    & \s{4}      & F  \\
          \bottomrule
        \end{tabular}
      \end{minipage}
    \end{center}

    By inverting the dominance frontier sets, we arrive at the control
    dependence graph below.
    \begin{center}
      \begin{minipage}[b]{.65\linewidth}
        \begin{center}
          \begin{tikzpicture}[node distance=1.4 and .8]
            \node[terminal] (E) {entry\todo{note the difference from above}};
            \node[block, below left=1.4 and .4 of S] (2) {2};
            \node[block, left=of 2] (1) {1};
            % \node[terminal, left=of 1] (E) {entry};
            \node[block, right=of 2] (3) {3};
            \node[block, right=of 3] (4) {4};

            \node[block, below=of 4] (8) {8};
            \node[block, left=of 8] (6) {6};
            \node[block, left=of 6] (5) {5};
            \node[block, right=of 8] (15) {15};
            \node[block, right=of 15] (23) {23};

            \node[block, below=of 8] (13) {13};
            \node[block, left=of 13] (12) {12};
            \node[block, left=of 12] (10) {10};
            \node[block, left=of 10] (9) {9};

            \node[block, below=of 15] (16) {16};
            \node[block, right=of 16] (17) {17};

            \node[block, below left=of 17] (18) {18};
            \node[block, below right=of 17] (20) {20};

            \path (E)
            % edge node[swap,very near end] {T} (E)
            edge node[near end, swap] {T} (1)
            edge node[near end, left=.1] {T} (2)
            edge node[near end, right=.1] {T} (3)
            edge node[near end] {T} (4);

            \path (4)
            edge node[swap,very near end] {T} (5)
            edge node[near end, left=.1] {T} (6)
            edge node[near end] {F} (8)
            edge node[near end, right=.1] {F} (15)
            edge node[very near end] {F} (23);

            \path (8)
            edge node[swap,near end] {T} (9)
            edge node[swap,near end, left=.1] {T} (10)
            edge node[near end,left=.1] {F} (12)
            edge node[near end] {F} (13);

            \path (15)
            edge node[near end] {T} (16)
            edge node[near end,right=.1] {T} (17)
            edge[loop left] node[below=.1] {T\todo{include? can't
                include if 15 dom 15}} (15);

            \path (17)
            edge node[near end, swap] {T} (18)
            edge node[near end] {F} (20);
          \end{tikzpicture}
        \end{center}
        Control dependence graph constructed using the dominance
        frontier method.\todo{need \emph{P}DG!}
      \end{minipage}
    \end{center}

\end{enumerate}

\end{document}

% vim: nowrap:guioptions+=b
