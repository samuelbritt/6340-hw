\documentclass{article}

\usepackage[T1]{fontenc}
\usepackage[osf]{mathpazo}
\usepackage{microtype}

\usepackage{amsmath}
\usepackage{amssymb}
\usepackage{amsfonts}
\usepackage{amsthm}
\usepackage{mathtools}
\usepackage{bm}
\usepackage{xcolor}
\usepackage{graphicx}
\usepackage[pdftex]{lscape}
\usepackage{lipsum}
\usepackage{tikz}
\usepackage{enumitem}
\usepackage[alsoload=binary]{siunitx}
\usepackage{booktabs}
\usepackage{multirow}
\usepackage{tabularx}
\usepackage{cancel}
\usepackage{fancyvrb}
\usepackage[section]{placeins}
\usepackage{flafter}
\usepackage{pdfpages}
\usepackage[hmargin=1.5in,vmargin=1in,showframe]{geometry}
\usepackage{url}
\usepackage{hyperref}

% algorithms
\let\oldgets\gets
\usepackage{clrscode3e}
\let\gets\oldgets
\usepackage{listings}

% no section numbers
\setcounter{secnumdepth}{-2}

% a bit more compact, and fixes spacing issues
\let\originalleft\left
\let\originalright\right
\renewcommand{\l}{\mathopen{}\mathclose\bgroup\originalleft}
\renewcommand{\r}{\aftergroup\egroup\originalright}

% commands
\renewcommand\vec[1]{\bm{#1}}
\renewcommand\epsilon{\varepsilon}
\newcommand\bit{\ensuremath{\set{0,1}}}
\newcommand\abs[1]{\l\vert #1 \r\vert}
\newcommand\set[1]{\ensuremath{\l\{#1\r\}}}
% redefine `\to` to use tikz arrows
\renewcommand\to{%
  \mathbin{\tikz[baseline=-.7ex] \draw[->] (0,0) -- +(.4,0);}%
}

% red todos
\usepackage{marginnote}
\usepackage{xifthen}% provides \isempty test
\renewcommand\marginfont{\color{red}}

% Arg 1: vertical shift, Arg 2: content
\newcommand\Todo[2][]{{\color{red}$^\dagger$}\marginnote{$^\dagger$#2}[#1]}

% wrapper for \Todo: can leave content and/or shift empty
\newcommand{\todo}[2][0pt]
  {%
    \ifthenelse{ \isempty{#2} }%
      {%
        \Todo[#1]{\textsc{todo}}%
      }%
      {%
        \Todo[#1]{#2}%
      }%
  }

\usetikzlibrary
{
  positioning,
  calc,
  intersections,
  arrows,
  shapes.misc,
  shapes.geometric,
}

\tikzset
  {
    ->,
    >=latex',
    shorten >=.5pt,
    node distance=.8cm and .8cm,
    fork/.style= {node distance=.8cm and #1},
    wide fork/.style={fork=1.6cm},
    block/.style=
      {
        draw,
        inner sep=.3em,
        rounded rectangle,
        minimum size=1.7em,
      },
    terminal/.style=
      {
        block,
        rectangle,
        text height=1.5ex,
        text depth=.25ex,
      },
    region/.style=
      {
        draw,
        trapezium,
      },
    edge label/.style=
      {
        auto=false,
        anchor=center,
        inner sep=.25ex,
        fill=white,
        font=\small
      },
    data edge/.style={red},
  }

% table customizaton
\newcommand\newrow{\\\addlinespace}

% center columns
\newcolumntype{M}[1]{>{\centering\arraybackslash}m{#1}}

% highlight
\newcommand\hi[1]{\textcolor{red}{#1}}

% data analysis
\newcommand\bottom{\ensuremath{\perp}}
\newcommand\union{\bigcup}
\newcommand\Kill{\texttt{KILL}}
\newcommand\Gen{\texttt{GEN}}
\newcommand\varfamily{\ttfamily}  % use in tables
\newcommand\var[1]{\text{\varfamily #1}}
\newcommand\blk[1]{\text{#1}}
\DeclareMathOperator\pdom{pdom}
\DeclareMathOperator\dom{dom}

\usepackage{filecontents}
\begin{filecontents*}{proc-M}
 1 begin M
 2   read i, j
 3   sum = 0
 4   while i <= 10 do
 5     call B
 6   endwhile
 7   print (i)
 8   call C
 9 end M
\end{filecontents*}


\begin{document}
\thispagestyle{empty}

\includepdf
  [
    pages=1,
    pagecommand=
      {
        \begin{tikzpicture}[remember picture, overlay]
          \node[text width=6.3in, text height=2.6in]
          at (current page.north east)
          {\large
            Sam Britt, Shriram Swaminathan,\\[1.5em]
            and Sivaramachandran Ganesan
          };
        \end{tikzpicture}
      }
  ]
  {Assignment5ProblemSet4}

\clearpage
\pagenumbering{arabic}

\noindent
Sam Britt                \hfill CS 6340      \\
Shriram Swaminathan      \hfill Assignment 5 \\
Sivaramachandran Ganesan \hfill Feb. 6, 2013

\begin{enumerate}
    \setlength\parskip{3ex}
  \item Augmented CDGs and PDGs for each of the three programs are
    shown below. Edges corresponding to data dependencies on the PDGs
    are shown in red, and are labeled by the variable they represent.

    \begin{minipage}[b]{.45\linewidth}
      \centering
      \begin{tikzpicture}
        \node[terminal] (E) {\blk{entry}};
        \node[block, below=of E] (2) {\blk{2}};
        \node[block, below=of 2] (3) {\blk{3}};
        \node[block, below=of 3] (4) {\blk{4}};
        \node[block, below right=of 4] (5) {\blk{5}};
        \node[block, below left=of 4] (7) {\blk{7}};
        \node[block, below=of 7] (8) {\blk{8}};
        \node[terminal, below=of 8] (X) {\blk{exit}};
        \node[terminal] (S) at ($(E)!.5!(X) + (-1.2,0)$) {\blk{start}};

        \draw (E) -- (2);
        \draw (2) -- (3);
        \draw (3) -- (4);
        \draw (4) -- node[edge label] {F} (7);
        \draw (4) -- node[edge label] {T} (5);
        \draw (5) |- (4);
        \draw (7) -- (8);
        \draw (8) -- (X);

        \draw (S) |- node[edge label,near start] {T} (E);
        \draw (S) |- node[edge label,near start] {F} (X);
      \end{tikzpicture}

      The augmented CFG for program M.
    \end{minipage}
    %
    \begin{minipage}[b]{.55\linewidth}
      \centering
      \begin{tikzpicture}[
          % on grid,
          % node distance=1.4 and 1.4,
        ]
        \node[terminal] (S) {\blk{start}};
        \node[block, below=of S] (3) {\blk{3}};
        \node[block, left=of 3] (2) {\blk{2}};
        \node[terminal, left=of 2] (E) {\blk{entry}};
        \node[block, right=of 3] (4) {\blk{4}};
        \node[terminal, right=of 4] (X) {\blk{exit}};

        \node[block, below left=of 4] (5) {\blk{5}};
        \node[block, below=of 4] (7) {\blk{7}};
        \node[block, right=of 7] (8) {\blk{8}};

        % This construction allows all labels to be aligned
        % vertically.
        % 1. Find the "anchor" point; here it is a coordinate halfway
        % between S and 3.
        \path (S) -- coordinate (anchor) (3);
        % 2. Draw a path as wide as necessary horizontally through the
        % anchor point
        \path[name path=anchor-horiz] (E|-anchor) -- (X|-anchor);
        \foreach \n/\val in {E/T, 2/T, 3/T, 4/T, X/F}
        {
          % 3. Draw the needed edges, making sure to name the path.
          \draw[name path=edge] (S) -- (\n);
          % 4. Finally, drop the node at the intersection of the
          % horizontal path and the edge just drawn.
          \node
            [
              edge label,
              name intersections={of=anchor-horiz and edge}
            ]
            at (intersection-1) {\val};
        }

        \path (4) -- coordinate (anchor) (7);
        \path[name path=anchor-horiz] (5|-anchor) -- (8|-anchor);
        \foreach \n/\val in {5/T, 7/F, 8/F}
        {
          \draw[name path=edge] (4) -- (\n);
          \node
            [
              edge label,
              name intersections={of=anchor-horiz and edge}
            ]
            at (intersection-1) {\val};
        }

        \begin{scope}[data edge]
          \draw[out=225, in=225, looseness=1.3]
            (2) to node[edge label] {\var{i}} (7);
        \end{scope}
      \end{tikzpicture}

      The PDG for program M.
    \end{minipage}

    \begin{minipage}[b]{.45\linewidth}
      \centering
      \begin{tikzpicture}
        \node[terminal] (E) {\blk{entry}};
        \node[block, below=of E] (11) {\blk{11}};
        \node[block, right=of 11] (12) {\blk{12}};
        \node[block, below=of 11] (14) {\blk{14}};
        \node[block, below=of 14] (15) {\blk{15}};
        \node[terminal, below=of 15] (X) {\blk{exit}};
        \node[terminal] at ($(E)!.5!(X) + (-1,0)$) (S) {\blk{start}};

        \draw (E) -- (11);
        \draw (11) -- node[edge label] {T} (12);
        \draw (12) -- (14);
        \draw (11) -- node[edge label] {F} (14);
        \draw (14) -- (15);
        \draw (15) -- (X);
        \draw (S) |- node[edge label, near start] {T} (E);
        \draw (S) |- node[edge label, near start] {F} (X);
      \end{tikzpicture}

      The augmented CFG for procedure B.
    \end{minipage}
    %
    \begin{minipage}[b]{.55\linewidth}
      \centering
      \begin{tikzpicture}
        \node[terminal] (S) {\blk{start}};
        \node[block, below=of S] (14) {\blk{14}};
        \node[block, left=of 14] (11) {\blk{11}};
        \node[terminal, left=of 11] (E) {\blk{entry}};
        \node[block, right=of 14] (15) {\blk{15}};
        \node[terminal, right=of 15] (X) {\blk{exit}};

        \node[block, below=of 11] (12) {\blk{12}};

        \path (S) -- coordinate (anchor) (14);
        \path[name path=anchor-horiz] (E|-anchor) -- (X|-anchor);
        \foreach \n/\val in {E/T, 11/T, 14/T, 15/T, X/F}
        {
          \draw[name path=edge] (S) -- (\n);
          \node
            [
              edge label,
              name intersections={of=anchor-horiz and edge}
            ]
            at (intersection-1) {\val};
        }

        \draw (11) -- node[edge label] {12} (12);

        \begin{scope}[data edge]
          
        \end{scope}
      \end{tikzpicture}

      The PDG for procedure B.\todo{data edges?}
    \end{minipage}

    \begin{minipage}[b]{.45\linewidth}
      \centering
      \begin{tikzpicture}%[on grid, node distance=1 and 1]
        \node[terminal] (E) {\blk{entry}};
        \node[block, below=of E] (18) {\blk{18}};
        \node[block, below right=of 18] (19) {\blk{19}};
        \node[block, below=of 19] (20) {\blk{20}};
        % \node[terminal, below left=of 20] (X) {\blk{exit}};
        \coordinate[below left=of 20] (aux) {};
        \node[terminal] at (aux-|18) (X) {\blk{exit}};
        \node[terminal] at ($(E)!.5!(X) + (-1,0)$) (S) {\blk{start}};

        \draw (E) -- (18);
        \draw (18) -- node[edge label] {T} (19);
        \draw (19) -- (20);
        \draw (20) -- (X);
        \draw (18) -- node[edge label] {F} (X);

        \draw (S) |- node[near start, edge label] {T} (E);
        \draw (S) |- node[near start, edge label] {F} (X);
      \end{tikzpicture}

      The augmented CFG for procedure C.
    \end{minipage}
    %
    \begin{minipage}[b]{.55\linewidth}
      \centering
      \begin{tikzpicture}
        \node[terminal] (S) {\blk{start}};
        \node[terminal, below left=of S] (E) {\blk{entry}};
        \node[block, below=of S] (18) {\blk{18}};
        \node[terminal, below right=of S] (X) {\blk{exit}};
        \node[block, below left=of 18] (19) {\blk{19}};
        \node[block, below right=of 18] (20) {\blk{20}};
        
        \path (S) -- coordinate (anchor) (18);
        \path[name path=anchor-horiz] (E|-anchor) -- (X|-anchor);
        \foreach \n/\val in {E/T, 18/T, X/F}
        {
          \draw[name path=edge] (S) -- (\n);
          \node
            [
              edge label,
              name intersections={of=anchor-horiz and edge}
            ]
            at (intersection-1) {\val};
        }

        \draw (18) -- node[edge label] {T} (19);
        \draw (18) -- node[edge label] {T} (20);

        \begin{scope}[data edge]
          \draw[bend right] (20) to node[edge label] {\var{j}} (18);
          \draw[bend left]  (20) to node[edge label] {\var{j}} (19);
        \end{scope}
      \end{tikzpicture}

      The PDG for procedure C.
    \end{minipage}
\end{enumerate}

\end{document}

% vim: nowrap:guioptions+=b
