\documentclass{article}

\usepackage[T1]{fontenc}
\usepackage[osf]{mathpazo}
\usepackage{microtype}

\usepackage{amsmath}
\usepackage{amssymb}
\usepackage{amsfonts}
\usepackage{amsthm}
\usepackage{mathtools}
\usepackage{bm}
\usepackage{xcolor}
\usepackage{graphicx}
\usepackage[pdftex]{lscape}
\usepackage{lipsum}
\usepackage{tikz}
\usepackage{enumitem}
\usepackage[alsoload=binary]{siunitx}
\usepackage{booktabs}
\usepackage{multirow}
\usepackage{tabularx}
\usepackage{cancel}
\usepackage{fancyvrb}
\usepackage[section]{placeins}
\usepackage{flafter}
\usepackage{pdfpages}
\usepackage[hmargin=1.5in,vmargin=1in]{geometry}
\usepackage{url}
\usepackage{hyperref}

% algorithms
\let\oldgets\gets
\usepackage{clrscode3e}
\let\gets\oldgets
\usepackage{listings}

% no section numbers
\setcounter{secnumdepth}{-2}

% a bit more compact, and fixes spacing issues
\let\originalleft\left
\let\originalright\right
\renewcommand{\l}{\mathopen{}\mathclose\bgroup\originalleft}
\renewcommand{\r}{\aftergroup\egroup\originalright}

% commands
\renewcommand\vec[1]{\bm{#1}}
\renewcommand\epsilon{\varepsilon}
\newcommand\bit{\ensuremath{\set{0,1}}}
\newcommand\abs[1]{\l\vert #1 \r\vert}
\newcommand\set[1]{\ensuremath{\l\{#1\r\}}}
% redefine `\to` to use tikz arrows
\renewcommand\to{%
  \mathbin{\tikz[baseline=-.7ex] \draw[->] (0,0) -- +(.4,0);}%
}

% red todos
\usepackage{marginnote}
\usepackage{xifthen}% provides \isempty test
\renewcommand\marginfont{\color{red}\footnotesize}

% Arg 1: vertical shift, Arg 2: content
\newcommand\Todo[2][]{{\color{red}$^\dagger$}\marginnote{$^\dagger$#2}[#1]}

% wrapper for \Todo: can leave content and/or shift empty
\newcommand{\todo}[2][0pt]
  {%
    \ifthenelse{ \isempty{#2} }%
      {%
        \Todo[#1]{\textsc{todo}}%
      }%
      {%
        \Todo[#1]{#2}%
      }%
  }

\usetikzlibrary
{
  positioning,
  calc,
  intersections,
  backgrounds,
  arrows,
  shapes.misc,
  shapes.geometric,
}

\tikzset
  {
    ->,
    >=latex',
    shorten >=.5pt,
    node distance=.8cm and .8cm,
    fork/.style= {node distance=.8cm and #1},
    wide fork/.style={fork=1.6cm},
    block/.style=
      {
        draw,
        inner sep=.3em,
        rounded rectangle,
        text height=1.3ex,
        text depth=.2ex,
        minimum size=3.6ex,
      },
    terminal/.style=
      {
        block,
        rectangle,
        text height=1.5ex,
        text depth=.25ex,
      },
    region/.style=
      {
        draw,
        trapezium,
      },
    selected/.style={fill=gray!30},
    edge label/.style=
      {
        auto=false,
        anchor=mid,
        fill=white,
        font=\footnotesize,
        inner sep=.25ex,
      },
    data edge/.style={red},
    call edge/.style={dashed},
    param edge/.style={blue},
    trans edge/.style={green!50!black},
    inline center/.style={baseline=-.6ex},
  }

\newcommand\SampleEdge[1]{%
  \tikz[inline center] \draw[#1] (0,0) -- (.7,0);%
}

% table customizaton
\newcommand\newrow{\\\addlinespace}

% center columns
\newcolumntype{M}[1]{>{\centering\arraybackslash}m{#1}}

% Centers giant figures, tables, etc, on the page, even within nested
% enumerates/itemize environments
\makeatletter
\newcommand\BigCenter[1]{\par\noindent\hspace*{-\@totalleftmargin}\makebox[\textwidth]{#1}}
\makeatother

% highlight
\newcommand\hi[1]{\textcolor{red}{#1}}

% data analysis
\newcommand\bottom{\ensuremath{\perp}}
\newcommand\union{\bigcup}
\newcommand\Kill{\texttt{KILL}}
\newcommand\Gen{\texttt{GEN}}
\newcommand\codefamily{\ttfamily}  % use in tables
\newcommand\code[1]{\text{\codefamily #1}}
\newcommand\blk[1]{\text{#1}}
\DeclareMathOperator\pdom{pdom}
\DeclareMathOperator\dom{dom}

\begin{document}
\thispagestyle{empty}

\includepdf
  [
    pages=1,
    pagecommand=
      {
        \begin{tikzpicture}[remember picture, overlay]
          \node[text width=6.3in, text height=2.6in]
          at (current page.north east)
          {\large
            Sam Britt, Shriram Swaminathan,\\[1.5em]
            and Sivaramachandran Ganesan
          };
        \end{tikzpicture}
      }
  ]
  {Assignment5ProblemSet4}

\clearpage
\pagenumbering{arabic}

\noindent
Sam Britt                \hfill CS 6340      \\
Shriram Swaminathan      \hfill Assignment 5 \\
Sivaramachandran Ganesan \hfill Feb. 6, 2013

\begin{enumerate}
    \setlength\parskip{3ex}
  \item CFGs and PDGs for each of the three programs are
    shown below. Edges corresponding to data dependencies on the PDGs
    are shown in red (\SampleEdge{data edge}), and are labeled by the
    variable they represent. Formal- and actual-in and out definition
    nodes are not included, so there are very few data dependencies.

    \begin{minipage}[b]{.5\linewidth}
      \centering
      \begin{tikzpicture}%[node distance=.5 and 1]
        \node[terminal] (E) {\blk{entry}};
        \node[block, right=of E] (2) {\blk{2}};
        \node[block, right=of 2] (3) {\blk{3}};
        \node[block, right=of 3] (4) {\blk{4}};
        \node[block, below right=of 4] (5) {\blk{5}};
        \node[block, below left=of 4] (7) {\blk{7}};
        \node[block, below=of 7] (8) {\blk{8}};
        \node[terminal, below=of 8] (X) {\blk{exit}};

        \draw (E) -- (2);
        \draw (2) -- (3);
        \draw (3) -- (4);
        \draw (4) -- node[edge label] {F} (7);
        \draw (4) -- node[edge label] {T} (5);
        \draw (5) |- (4);
        \draw (7) -- (8);
        \draw (8) -- (X);
      \end{tikzpicture}

      The CFG for program $M$.
    \end{minipage}
    %
    \begin{minipage}[b]{.5\linewidth}
      \centering
      \begin{tikzpicture}[
          % on grid,
          % node distance=1.4 and 1.4,
        ]
        % \draw[help lines, -, step=.5] (-3.5,-3.5) grid      (3.25,.5);
        \path[use as bounding box]    (-3.5,-3.5) rectangle (3.25,.5);

        \node[terminal] (E) {\blk{entry}};
        \node[block, below=of E] (4) {\blk{4}};
        \node[block, left=of 4] (3) {\blk{3}};
        \node[block, left=of 3] (2) {\blk{2}};

        \node[block, right=of 4] (7) {\blk{7}};
        \node[block, right=of 7] (8) {\blk{8}};
        \node[block, below=of 4] (5) {\blk{5}};

        % This construction allows all labels to be aligned
        % vertically.
        % 1. Find the "anchor" point; here it is a coordinate halfway
        % between E and 4.
        \path (E) -- coordinate (anchor) (4);
        % 2. Draw a path as wide as necessary horizontally through the
        % anchor point
        \path[name path=anchor-horiz] (2|-anchor) -- (8|-anchor);
        \foreach \n/\val in {2/T, 3/T, 4/T, 7/T, 8/T}
        {
          % 3. Draw the needed edges, making sure to name the path.
          \draw[name path=edge] (E) -- (\n);
          % 4. Finally, drop the node at the intersection of the
          % horizontal path and the edge just drawn.
          \node
            [
              edge label,
              name intersections={of=anchor-horiz and edge}
            ]
            at (intersection-1) {\val};
        }

        \draw[loop left] (4) to node[edge label, above] {T} (4);
        \draw (4) -- node[edge label] {T} (5);

        \begin{scope}[data edge]
          \draw[out=225, in=-45, looseness=1.7]
            (2) to node[edge label] {\code{i}} (7);
          \draw[out=225, in=210, looseness=1.1]
            (2) to node[edge label] {\code{i}} (4);
        \end{scope}
      \end{tikzpicture}

      The PDG for program $M$.
    \end{minipage}

    \begin{minipage}[b]{.5\linewidth}
      \centering
      \begin{tikzpicture}
        \node[terminal] (E) {\blk{entry}};
        \node[block, right=of E] (11) {\blk{11}};
        \node[block, right=of 11] (12) {\blk{12}};
        \node[block, below=of 11] (14) {\blk{14}};
        \node[block, below=of 14] (15) {\blk{15}};
        \node[terminal, below=of 15] (X) {\blk{exit}};

        \draw (E) -- (11);
        \draw (11) -- node[edge label] {T} (12);
        \draw (12) -- (14);
        \draw (11) -- node[edge label, anchor=base] {F} (14);
        \draw (14) -- (15);
        \draw (15) -- (X);
      \end{tikzpicture}

      The CFG for procedure $B$.
    \end{minipage}
    %
    \begin{minipage}[b]{.5\linewidth}
      \centering
      \begin{tikzpicture}
        \node[terminal] (E) {\blk{entry}};
        \node[block, below=of E] (14) {\blk{14}};
        \node[block, left=of 14] (11) {\blk{11}};
        \node[block, right=of 14] (15) {\blk{15}};

        \node[block, below=of 11] (12) {\blk{12}};

        \path (E) -- coordinate (anchor) (14);
        \path[name path=anchor-horiz] (11|-anchor) -- (15|-anchor);
        \foreach \n/\val in {11/T, 14/T, 15/T}
        {
          \draw[name path=edge] (E) -- (\n);
          \node
            [
              edge label,
              name intersections={of=anchor-horiz and edge}
            ]
            at (intersection-1) {\val};
        }

        \draw (11) -- node[edge label] {T} (12);

        \begin{scope}[data edge]
          
        \end{scope}
      \end{tikzpicture}

      The PDG for procedure $B$.
    \end{minipage}

    \begin{minipage}[b]{.5\linewidth}
      \centering
      \begin{tikzpicture}%[on grid, node distance=1 and 1]
        \node[terminal] (E) {\blk{entry}};
        \node[block, right=of E] (18) {\blk{18}};
        \node[block, below right=of 18] (19) {\blk{19}};
        \node[block, below=of 19] (20) {\blk{20}};
        \coordinate[below=of 20] (aux) {};
        \node[terminal] at (aux-|18) (X) {\blk{exit}};

        \draw (E) -- (18);
        \draw (18) -- node[edge label] {T} (19);
        \draw (19) -- (20);
        \draw (20) -- (X);
        \draw (18) -- node[edge label] {F} (X);
      \end{tikzpicture}

      The CFG for procedure $C$.
    \end{minipage}
    %
    \begin{minipage}[b]{.5\linewidth}
      \centering
      \begin{tikzpicture}
        \node[terminal] (E) {\blk{entry}};
        \node[block, below=of E] (18) {\blk{18}};
        \node[block, below left=of 18] (19) {\blk{19}};
        \node[block, below right=of 18] (20) {\blk{20}};
        
        \draw  (E) -- node[edge label] {T} (18);
        \draw (18) -- node[edge label] {T} (19);
        \draw (18) -- node[edge label] {T} (20);

        \begin{scope}[data edge]
          % \draw[bend right] (20) to node[edge label] {\code{j}} (18);
          % \draw[bend left]  (20) to node[edge label] {\code{j}} (19);
        \end{scope}
      \end{tikzpicture}

      The PDG for procedure $C$.
    \end{minipage}

    \newpage
  \item In the following SDG,
    control dependencies are in solid black (\SampleEdge{}),
    call edges are in dashed black (\SampleEdge{call edge}),
    data dependencies are in red (\SampleEdge{data edge}),
    parameter in and out edges are in blue (\SampleEdge{param edge}),
    and transitive flow dependence edges are in green (\SampleEdge{trans edge}).

    \newcommand\SDG{%
      \BigCenter{%
        \begin{tikzpicture}[node distance=1 and .5, remember picture]
          \path[use as bounding box]      (-7,-17) rectangle (12.5,0.5);
          % \draw[-, help lines, step=.5cm] (-7,-17) grid      (12.5,0.5);

          %%% NODES

          % Program M
          \node[terminal] (EM) {\blk{entry $M$}};
          \node[block, draw=none, below=of EM] (aux1) {};
          \node[block, draw=none, left=of aux1] (aux2) {};
          \node[block] at ($(aux1)!.5!(aux2)$) (3) {\blk{3}};
          \node[block, left=of 3] (2) {\blk{2}};

          \node[block, right=of 3] (4) {\blk{4}};
          \node[block, right=of 4] (7) {\blk{7}};

          % call to M to B
          \node[block, below left=1.5 and 2 of 4] (5) {\blk{5. \code{call B}}};
          \coordinate[left=-.2 of 5] (MB-aux-in);
          \coordinate[right=-.2 of 5] (MB-aux-out);
          \foreach \below/\var in {aux/i, i/j, j/sum}
            {
              \node[block, below=1 of MB-\below-in.east, anchor=east]
                (MB-\var-in) {\code{\var\_in = \var}};

              \node[block, below=1 of MB-\below-out.west, anchor=west]
                (MB-\var-out) {\code{\var\ = \var\_out}};
            }

          % call to M to C
          \node[block, right=4.5 of 7] (8) {\blk{8. \code{call C}}};
          \coordinate[left=-.3 of 8] (MC-aux-in);
          \coordinate[right=-.3 of 8] (MC-aux-out);
          \foreach \below/\var in {aux/j, j/sum}
            {
              \node[block, below=1 of MC-\below-in.east, anchor=east]
                (MC-\var-in) {\code{\var\_in = \var}};

              \node[block, below=1 of MC-\below-out.west, anchor=west]
                (MC-\var-out) {\code{\var\ = \var\_out}};
            }

          % Procedure B
          \node[terminal, below=5 of 5] (EB) {\blk{entry $B$}};

          \coordinate[ left=1.2 of EB] (B-aux-in);
          \coordinate[right=1.2 of EB] (B-aux-out);
          \foreach \below/\var in {aux/i, i/j, j/sum}
            {
              \node[block, below=1 of B-\below-in.east, anchor=east]
                (B-\var-in) {\code{\var\ = \var\_in}};

              \node[block, below=1 of B-\below-out.west, anchor=west]
                (B-\var-out) {\code{\var\_out = \var}};
            }

          \coordinate[below=1.5 of EB] (aux);
          \node[block, right=of aux] (15) {\blk{15}};
          \node[block, left=of aux] (11) {\blk{11}};
          \node[block, below=of 11] (12) {\blk{12}};

          % call to C
          \node[block, below=2.25 of aux] (14) {\blk{14. \code{call C}}};
          \coordinate[left=-.3 of 14] (BC-aux-in);
          \coordinate[right=-.3 of 14] (BC-aux-out);
          \foreach \below/\var in {aux/j, j/sum}
            {
              \node[block, below=1 of BC-\below-in.east, anchor=east]
                (BC-\var-in) {\code{\var\_in = \var}};

              \node[block, below=1 of BC-\below-out.west, anchor=west]
                (BC-\var-out) {\code{\var\ = \var\_out}};
            }

          % Procedure C
          \node[terminal] at (8|-EB) (EC) {\blk{entry $C$}};
          \node[block, below=of EC] (18) {\blk{18}};
          \node[block, below left=of 18] (19) {\blk{19}};
          \node[block, below right=of 18] (20) {\blk{20}};

          \coordinate[left=.2 of EC] (C-aux-in);
          \coordinate[right=.2 of EC] (C-aux-out);
          \foreach \below/\var in {aux/j, j/sum}
            {
              \node[block, below=1 of C-\below-in.east, anchor=east]
                (C-\var-in) {\code{\var\ = \var\_in}};

              \node[block, below=1 of C-\below-out.west, anchor=west]
                (C-\var-out) {\code{\var\_out = \var}};
            }


          %%% EDGES

          \begin{scope}[call edge]
            \draw (5) -- (EB);
            \draw (8) -- (EC);
            \draw (14) -| ($(EB)!.58!(EC)$) |- (EC);
          \end{scope}

          \begin{scope} % Control Edges
            \path (EM) -- coordinate (anchor) (4);
            \path[name path=anchor-horiz] (2|-anchor) -- (8|-anchor);
            \foreach \n/\val in {2/T, 3/T, 4/T, 7/T, 8/T}
              {
                \draw[name path=edge] (EM) -- (\n);
                \node
                  [
                    edge label,
                    name intersections={of=anchor-horiz and edge}
                  ]
                  at (intersection-1) {\val};
              }

            \draw[loop left] (4) to node[edge label, above] {T} (4);
            \draw (4) -- node[edge label] {T} (5);

            \foreach \var in {i, j, sum}
              {
                \draw (5) -- (MB-\var-in.5);
                \draw (5) -- (MB-\var-out.180 - 5);
              }
            \foreach \var in {j, sum}
              {
                \draw (8) -- (MC-\var-in.10);
                \draw (8) -- (MC-\var-out.180 - 10);
              }

            % Procedure B
            \draw (EB) -- node[edge label] {T} (11);
            \draw (EB) -- node[edge label] {T} (14);
            \draw (EB) -- node[edge label] {T} (15);
            \draw (11) -- node[edge label] {T} (12);

            \foreach \var in {i, j, sum}
              {
                \draw (EB) -- (B-\var-in.10);
                \draw (EB) -- (B-\var-out.180 - 10);
              }

            \foreach \var in {j, sum}
              {
                \draw (14) -- (BC-\var-in.10);
                \draw (14) -- (BC-\var-out.170);
              }

            % Procedure C
            \foreach \var in {j, sum}
              {
                \draw (EC) -- (C-\var-in.10);
                \draw (EC) -- (C-\var-out.170);
              }
            \draw (EC) -- node[edge label] {T} (18);
            \draw (18) -- node[edge label] {T} (19);
            \draw (18) -- node[edge label] {T} (20);
          \end{scope}

          \begin{scope}[trans edge]
            % M calls B
            \draw (MB-i-in) to (MB-i-out);
            \draw (MB-j-in) to (MB-j-out);
            \draw (MB-sum-in) to (MB-sum-out);

            % M calls C
            \draw (MC-j-in) to (MC-j-out);
            \draw (MC-sum-in) to (MC-sum-out);

            % B calls C
            \draw (BC-j-in) to (BC-j-out);
            \draw (BC-sum-in) to (BC-sum-out);
          \end{scope}

          \begin{scope}[param edge, looseness=.75]
            % M calls B
            \foreach \node in {i, j, sum}
              {
                \draw[out=180+10,in=180-30] (MB-\node-in) to (B-\node-in.north west);
                \draw[out=30,in=-10] (B-\node-out) to (MB-\node-out);
              }

            % M calls C
            \foreach \node in {j, sum}
              {
                \draw[out=180,in=180-30] (MC-\node-in) to (C-\node-in);
                \draw[out=30,in=0] (C-\node-out) to (MC-\node-out);
              }

            % B calls C
            \draw[out=-25,in=-25] (BC-j-in) to (C-j-in);
            \draw[out=-45,in=-45] (BC-sum-in) to (C-sum-in);
            \draw[out=0,in=0, looseness=1.3] (C-j-out) to (BC-j-out);
            \draw[out=0,in=0, looseness=1.3] (C-sum-out) to (BC-sum-out);
          \end{scope}

          \begin{scope}[data edge]
            % Program M
            % intra-procedural
            \draw[bend right=35] (2) to node[edge label, pos=.6] {\code{i}} (4);
            \draw[bend right=35] (2) to node[edge label, pos=.6] {\code{i}} (7);

            % call to B
            \draw[out=180+35, in=90, looseness=1]
              (2) to node[edge label] {\code{i}} (MB-i-in);
            \draw[out=180+35, in=180-25, looseness=1.8]
              (2) to node[edge label] {\code{j}} (MB-j-in);
            \draw[out=180-35, in=180-25, looseness=1.6]
              (3) to node[edge label] {\code{sum}} (MB-sum-in);

            % return from B
            \draw[out=0, in=-45]
              (MB-i-out) to node[edge label] {\code{i}} (4);
            \draw[out=0, in=-45]
              (MB-i-out) to node[edge label] {\code{i}} (7);

            % loop carried edges
            \draw[bend left]
              (MB-i-out) to node[edge label, pos=.45] {\code{i}} (MB-i-in);
            \draw[bend left]
              (MB-j-out) to node[edge label, pos=.45] {\code{j}} (MB-j-in);
            \draw[bend left]
              (MB-sum-out) to node[edge label] {\code{sum}} (MB-sum-in);

            % call to C
            \draw[out=0, in=180]
              (MB-j-out) to node[edge label, near start] {\code{j}} (MC-j-in);
            \draw[out=0, in=180]
              (MB-sum-out) to node[edge label, near start] {\code{sum}} (MC-sum-in);

            % Procedure B
            \draw[bend right]
              (B-sum-in) to node[edge label] {\code{sum}} (11);
            \draw[out=0, in=160, looseness=1.7]
              (B-i-in) to node[edge label, near end] {\code{i}} (15);
            \draw[out=0, in=180]
              (15) to node[edge label] {\code{i}} (B-i-out);

            % call to C
            \draw[out=180, in=180-25, looseness=1.8]
              (B-j-in) to node[edge label, near end] {\code{j}} (BC-j-in);
            \draw[out=180, in=180-15, looseness=1]
              (B-sum-in) to node[edge label] {\code{sum}} (BC-sum-in);
            \draw[in=0, out=25, looseness=1.8]
              (BC-j-out) to node[edge label, pos=.35] {\code{j}} (B-j-out);
            \draw[in=0, out=15, looseness=1]
              (BC-sum-out) to node[edge label, pos=.45] {\code{sum}} (B-sum-out);

            % Procedure M
            \draw
              (C-j-in) to node[edge label, pos=.61] {\code{j}} (18);
            \draw[out=180, in=225, looseness=2]
              (C-j-in) to node[edge label] {\code{j}} (19);
            \draw[out=225, in=225, looseness=1.3]
              (C-sum-in) to node[edge label, near start] {\code{sum}} (19);
            \draw
              (C-j-in) to node[edge label, pos=.65] {\code{j}} (C-j-out);
            \draw [out=-45, in=0, looseness=2]
              (20) to node[edge label] {\code{j}} (C-j-out);
            \draw[bend right=25]
              (C-sum-in) to node[edge label,near end] {\code{sum}} (C-sum-out);
            \draw[out=-45, in=-45, looseness=1.3]
              (19) to node[edge label, near end] {\code{sum}} (C-sum-out);
          \end{scope}
        \end{tikzpicture}
      } % end BigCenter
    } % end SDG

    \SDG

    \newpage
  \item In the following, nodes included in the slice are shaded. Each
        slice was constructed using the two-phase method on the SDG as
        described in lecture. Although the slices were determined via
        the algorithm, intuitive justificiations for the inclusion of
        select nodes are added where appropriate.
    \begin{enumerate}
      \item The slice according to the criterion $\l< \blk{7},
        \code{i} \r>$. It is interesting that line~4 is included in the
        slice. While it does not directly affect the value of
        \code{i}, its execution does control whether the call to $B$
        on line~5 is executed, and procudre $B$ does define \code{i}.
        Therefore, line~4 must be included.
        \SDG
        \begin{tikzpicture}[remember picture, overlay]
          \foreach \style/\n/\txt in
            {
              block/2,
              block/4,
              block/5/5. \code{call B},
              block/7,
              block/15,
              block/MB-i-in/\code{i\_in = i},
              block/MB-i-out/\code{i = i\_out},
              block/B-i-in/\code{i = i\_in},
              block/B-i-out/\code{i\_out = i},
              terminal/EM/entry $M$,
              terminal/EB/entry $B$%
            }
            \node[\style, selected] at (\n) {\blk{\txt}};
        \end{tikzpicture}

        \newpage
      \item The slice according to the criterion $\l< \blk{15},
        \code{i} \r>$. It is perhaps not at first intuitive why the
        formal out definition of \code{i\_out} in procedure $B$ and the
        actual out definition of \code{i} in in the call to $B$ on
        line~5 of program $M$ are included in the slice. However,
        because $B$ is called on line~5 inside a loop, the formal and
        actual out definitions of \code{i} in $B$ and $M$ eventually
        determine the evaluation of the loop condition in line~4 in
        $M$, which controls the further evaluation of $B$ (and thus
        line~15) inside the loop. Therefore, the inclusion of the
        formal and actual out definitions is justified.
        \SDG
        \begin{tikzpicture}[remember picture, overlay]
          \foreach \style/\n/\txt in
            {
              terminal/EM/entry $M$,
              block/2,
              block/4,
              block/5/5. \code{call B},
              block/MB-i-in/\code{i\_in = i},
              block/MB-i-out/\code{i = i\_out},
              terminal/EB/entry $B$,
              block/B-i-in/\code{i = i\_in},
              block/B-i-out/\code{i\_out = i},
              block/15%
            }
            \node[\style, selected] at (\n) {\blk{\txt}};
        \end{tikzpicture}

        \newpage
      \item The slice according to the criterion $\l< \blk{18},
        \code{j} \r>$. Again, it is perhaps not clear why the formal
        out defininition of \code{j\_out} in procedure $C$, or indeed
        line~20, is included in the slice. But, since $C$ is perhaps
        called many times (because $B$ is called inside a loop on
        line~5, and $B$ calls $C$ on line~14), then the (potential)
        re-definition of \code{j} on line~20 and as a formal out
        parameter will affect the use of \code{j} on line~18 the next
        time $C$ is called. Similarly, statements dealing with the
        definition of \code{i}, for example line~15, are included,
        because they control when $B$ (and therefore $C$) gets called
        on line~5. However, the actual out definition of \code{j} on
        the call to $C$ in line~8 of program $M$ is \emph{not}
        included in the slice, because by the time the call on line~8
        returns, there is no way for $C$ to be re-entered.  Therefore,
        the use of \code{j} line~18 is not affected by the actual out
        definition of \code{j} on line~8. 

        \SDG
        \begin{tikzpicture}[remember picture, overlay]
          \foreach \style/\n/\txt in
            {
              terminal/EC/entry $C$,
              block/18,
              block/20,
              block/C-j-in/\code{j = j\_in},
              block/C-j-out/\code{j\_out = j},
              terminal/EM/entry $M$,
              block/2,
              block/4,
              block/MB-j-in/\code{j\_in = j},
              block/MB-j-out/\code{j = j\_out},
              block/MB-i-in/\code{i\_in = i},
              block/MB-i-out/\code{i = i\_out},
              block/5/5. \code{call B},
              block/MC-j-in/\code{j\_in = j},
              block/8/8. \code{call C},
              terminal/EB/entry $B$,
              block/B-i-out/\code{i\_out = i},
              block/B-j-out/\code{j\_out = j},
              block/15,
              block/B-i-in/\code{i = i\_in},
              block/B-j-in/\code{j = j\_in},
              block/BC-j-in/\code{j\_in = j},
              block/BC-j-out/\code{j = j\_out},
              block/14/14. \code{call C}%
            }
            \node[\style, selected] at (\n) {\blk{\txt}};
        \end{tikzpicture}
    \end{enumerate}
\end{enumerate}

\end{document}

% vim: nowrap:guioptions+=b
