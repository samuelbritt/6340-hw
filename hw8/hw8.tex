\documentclass{article}

\usepackage[T1]{fontenc}
\usepackage[osf]{mathpazo}
\usepackage{microtype}

\usepackage{amsmath}
\usepackage{amssymb}
\usepackage{amsfonts}
\usepackage{amsthm}
\usepackage{mathtools}
\usepackage{bm}
\usepackage{xcolor}
\usepackage{graphicx}
\usepackage[pdftex]{lscape}
\usepackage{lipsum}
\usepackage{tikz}
\usepackage{enumitem}
\usepackage[alsoload=binary]{siunitx}
\usepackage{booktabs}
\usepackage{multirow}
\usepackage{tabularx}
\usepackage{cancel}
\usepackage{fancyvrb}
\usepackage[section]{placeins}
\usepackage{flafter}
\usepackage{pdfpages}
\usepackage[hmargin=1.5in,vmargin=1in]{geometry}
\usepackage{url}
\usepackage{hyperref}

% algorithms
\let\oldgets\gets
\usepackage{clrscode3e}
\let\gets\oldgets
\usepackage{listings}

% no section numbers
\setcounter{secnumdepth}{-2}

% a bit more compact, and fixes spacing issues
\let\originalleft\left
\let\originalright\right
\renewcommand{\l}{\mathopen{}\mathclose\bgroup\originalleft}
\renewcommand{\r}{\aftergroup\egroup\originalright}

% commands
\renewcommand\vec[1]{\bm{#1}}
\renewcommand\epsilon{\varepsilon}
\newcommand\bit{\ensuremath{\set{0,1}}}
\newcommand\abs[1]{\l\vert #1 \r\vert}
\newcommand\set[1]{\ensuremath{\l\{#1\r\}}}
% redefine `\to` to use tikz arrows
\renewcommand\to{%
  \mathbin{\tikz[baseline=-.7ex] \draw[->] (0,0) -- +(.4,0);}%
}

% red todos
\usepackage{marginnote}
\usepackage{xifthen}% provides \isempty test
\renewcommand\marginfont{\color{red}\footnotesize}

% Arg 1: vertical shift, Arg 2: content
\newcommand\Todo[2][]{{\color{red}$^\dagger$}\marginnote{$^\dagger$#2}[#1]}

% wrapper for \Todo: can leave content and/or shift empty
\newcommand{\todo}[2][0pt]
  {%
    \ifthenelse{ \isempty{#2} }%
      {%
        \Todo[#1]{\textsc{todo}}%
      }%
      {%
        \Todo[#1]{#2}%
      }%
  }

\usetikzlibrary
{
  positioning,
  calc,
  intersections,
  backgrounds,
  arrows,
  shapes.misc,
  shapes.geometric,
}

\tikzset
  {
    ->,
    >=latex',
    shorten >=.5pt,
    node distance=.8cm and .8cm,
    fork/.style= {node distance=.8cm and #1},
    wide fork/.style={fork=1.6cm},
    block/.style=
      {
        draw,
        inner sep=.3em,
        rounded rectangle,
        text height=1.3ex,
        text depth=.2ex,
        minimum size=3.6ex,
      },
    terminal/.style=
      {
        block,
        rectangle,
        text height=1.5ex,
        text depth=.25ex,
      },
    joint/.style={fill=black,circle,inner sep=0,minimum size=5pt},
    region/.style=
      {
        draw,
        trapezium,
      },
    selected/.style={fill=gray!30},
    edge label/.style=
      {
        auto=false,
        anchor=mid,
        fill=white,
        font=\footnotesize,
        inner sep=.25ex,
      },
    data edge/.style={red},
    call edge/.style={dashed},
    param edge/.style={blue},
    trans edge/.style={green!50!black},
    inline center/.style={baseline=-.6ex},
  }

\newcommand\SampleEdge[1]{%
  \tikz[inline center] \draw[#1] (0,0) -- (.7,0);%
}

% table customizaton
\newcommand\newrow{\\\addlinespace}

% center columns
\newcolumntype{M}[1]{>{\centering\arraybackslash}m{#1}}

% Centers giant figures, tables, etc, on the page, even within nested
% enumerates/itemize environments
\makeatletter
\newcommand\BigCenter[1]{\par\noindent\hspace*{-\@totalleftmargin}\makebox[\textwidth]{#1}}
\makeatother

% highlight
\newcommand\hi[1]{\textcolor{red}{#1}}

% data analysis
\newcommand\bottom{\ensuremath{\perp}}
\newcommand\union{\bigcup}
\newcommand\Kill{\texttt{KILL}}
\newcommand\Gen{\texttt{GEN}}
\newcommand\codefamily{\ttfamily}  % use in tables
\newcommand\code[1]{\text{\codefamily #1}}
\newcommand\blk[1]{\text{#1}}
\DeclareMathOperator\pdom{pdom}
\DeclareMathOperator\dom{dom}

\usepackage{bm}
\newcommand\coverage[2]{\textbf{\SI[detect-weight]{#1}{\percent} #2}}
\newcommand\covered{\ensuremath{\checkmark}}
\begin{document}
\thispagestyle{empty}

\includepdf
  [
    pages=1,
    pagecommand=
      {
        \begin{tikzpicture}[remember picture, overlay]
          \node[text width=6.3in, text height=2.6in]
          at (current page.north east)
          {\large
            Sam Britt, Shriram Swaminathan,\\[1.5em]
            and Sivaramachandran Ganesan
          };
        \end{tikzpicture}
      }
  ]
  {Assignment8ProblemSet5}

\clearpage
\pagenumbering{arabic}

\noindent
Sam Britt                \hfill CS 6340      \\
Shriram Swaminathan      \hfill Assignment 8 \\
Sivaramachandran Ganesan \hfill Mar. 12, 2013

Our test suite, along with the test results, is shown in
Table~\ref{tab:test_suite} below. We achieved \coverage{100}{statement
coverage} with this test suite; Table~\ref{tab:statement-coverage}
shows which tests cover each statement. Our test suite achieved
\coverage{65.8}{multiple condition coverage}; the coverage is
tabulated in Table~\ref{tab:condition-coverage}.  When considering
just the subset of conditions necessary for MC/DC coverage, our suite
achieves \coverage{92.6}{MC/DC coverage}. These data are tabulated in
Table~\ref{tab:MCDC-coverage}.

The CFG for \code{tritype.c} can be seen in Figure~\ref{fig:cfg}. The
cyclomatic complexity is 11, obtained by counting the number of
decision statements in the CFG and adding one. We identified ten
linearly independent paths from Table~\ref{tab:statement-coverage},
resulting in \coverage{90.9}{basis path coverage}.

The statement coverage results were obtained by using a debugger to
step through an execution of the program for each test case, while
marking which statements were executed. For multiple condition and
MC/DC coverage, we used a similar procedure: after identifying all the
truth vectors for each decision in the program, we stepped through
each test case in a debugger and marked which truth vector was
executed by each test. Some reasons for the low multiple condition
coverage are:
\begin{enumerate}
  \item The test suite does not have a test case with multiple
    negative inputs.
  \item The test suite does not have as many test cases for scalene
    output as required. Line 63 has three conditions and eight truth
    vectors; our test suite covers only two of them.
\end{enumerate}

The program has a binary operator bugs at lines~41 and~63.  At
line~41, the check should be for \code{k<=0} and not \code{k<0}.  This
bug was identified by test case~9 for which inputs $(\code{i},
\code{j}, \code{k})$ were $(1, 1, 0)$ and expected output was
``invalid,'' but the actual output was found to be ``isosceles.''
Similarly, at line~63, the condition should be \code{i+k<=j} and not
\code{i+k<j}. However, this bug was not identified by our test suite;
this is a result of the poor condition coverage of for the decision on
line~63 (see Table~\ref{tab:condition-coverage}), as well as not
properly testing the boundary conditions for violation of the triangle
inequality; e.g., the case where $\code{i} + \code{k} = \code{j}$ for
$\code{i}, \code{j}, \code{k} \ne 0$.

The program accepts invalid inputs like character strings instead of
strictly integer inputs. Test case~12 expects ``invalid'' as output,
but the execution resulted in ``isosceles.'' This illustrates that
input values are not validated properly by the program. However,
tests~10 and~11 both passed, even though they should not have. This is
because of the unpredictable nature of the bug---the compiler has left
some variables undefined, and the result coincidentally caused the
tests to pass.

\newcommand\pass{\textsc{pass}}
\newcommand\fail{\hi{\textsc{fail}}}
\begin{table}[htbp]
  \centering
  \begin{minipage}[t]{\linewidth}
    \centering
    \caption{Test Suite}
    \label{tab:test_suite}
    \BigCenter{
    \begin{tabular}{c l c l l}
      \toprule
      Test ID & Reason for test & Input: $(i, j, k)$  & Expected Output & Test Result\\
      \midrule
      1  & Violation of triangle inequality & $(10,  1,  1)$        & invalid      & \pass \\
      2  & Violation of triangle inequality & $( 1, 10,  1)$        & invalid      & \pass \\
      3  & Violation of triangle inequality & $( 1,  1, 10)$        & invalid      & \pass \\
      4  & Negative input                   & $(-1,  1,  1)$        & invalid      & \pass \\
      5  & Negative input                   & $( 1, -1,  1)$        & invalid      & \pass \\
      6  & Negative input                   & $( 1,  1, -1)$        & invalid      & \pass \\
      7  & Zero input                       & $( 0,  1,  1)$        & invalid      & \pass \\
      8  & Zero input                       & $( 1,  0,  1)$        & invalid      & \pass \\
      9  & Zero input                       & $( 1,  1,  0)$        & invalid      & \fail \\
      10 & Invalid input                    & $( \code{w},  1,  1)$ & invalid      & \pass \\
      11 & Invalid input                    & $( 1,  \code{w},  1)$ & invalid      & \pass \\
      12 & Invalid input                    & $( 1,  1,  \code{w})$ & invalid      & \fail \\
      13 & Equilateral                      & $( 1,  1,  1)$        & equilateral  & \pass \\
      14 & Scalene                          & $( 4,  3,  2)$        & scalene      & \pass \\
      15 & Isosceles                        & $( 2,  2,  3)$        & isosceles    & \pass \\
      16 & Isosceles                        & $( 2,  3,  2)$        & isosceles    & \pass \\
      17 & Isosceles                        & $( 3,  2,  2)$        & isosceles    & \pass \\
      \bottomrule
    \end{tabular}
    } % Bigcenter
  \end{minipage}
\end{table}

\begin{table}[htbp]
  \centering
  \begin{minipage}[t]{\linewidth}
    \centering
    \caption{Statement Coverage per Test}
    \label{tab:statement-coverage}
    \BigCenter{
    \begin{tabular}{c *{24}{c}}
      \toprule
      & \multicolumn{24}{c}{Statement line number} \\
      \cmidrule(lr){2-25}
      Test ID
              & 33       & 41       & 43       & 47       & 48       & 49       & 50       & 51       & 52       & 53       & 55       & 63       & 64       & 66       & 74       & 75       & 76       & 77       & 78       & 79       & 80       & 81       & 83       & 86            \\
      \midrule
      1       & \covered & \covered &          & \covered & \covered &          & \covered &          & \covered & \covered & \covered &          &          &          & \covered &          & \covered &          & \covered &          & \covered &           & \covered & \covered     \\
      2       & \covered & \covered &          & \covered & \covered &          & \covered & \covered & \covered &          & \covered &          &          &          & \covered &          & \covered &          & \covered &          & \covered &           & \covered & \covered     \\
      3       & \covered & \covered &          & \covered & \covered & \covered & \covered &          & \covered &          & \covered &          &          &          & \covered &          & \covered &          & \covered &          & \covered &           & \covered & \covered     \\
      4       & \covered & \covered & \covered &          &          &          &          &          &          &          &          &          &          &          &          &          &          &          &          &          &          &           &          & \covered     \\
      5       & \covered & \covered & \covered &          &          &          &          &          &          &          &          &          &          &          &          &          &          &          &          &          &          &           &          & \covered     \\
      6       & \covered & \covered & \covered &          &          &          &          &          &          &          &          &          &          &          &          &          &          &          &          &          &          &           &          & \covered     \\
      7       & \covered & \covered & \covered &          &          &          &          &          &          &          &          &          &          &          &          &          &          &          &          &          &          &           &          & \covered     \\
      8       & \covered & \covered & \covered &          &          &          &          &          &          &          &          &          &          &          &          &          &          &          &          &          &          &           &          & \covered     \\
      9       & \covered & \covered &          & \covered & \covered & \covered & \covered &          & \covered &          & \covered &          &          &          & \covered &          & \covered & \covered &          &          &          &           &          & \covered     \\
      10      & \covered & \covered & \covered &          &          &          &          &          &          &          &          &          &          &          &          &          &          &          &          &          &          &           &          & \covered     \\
      11      & \covered & \covered &          & \covered & \covered &          & \covered &          & \covered &          & \covered & \covered & \covered &          &          &          &          &          &          &          &          &           &          & \covered     \\
      12      & \covered & \covered &          & \covered & \covered & \covered & \covered &          & \covered &          & \covered &          &          &          & \covered &          & \covered & \covered &          &          &          &           &          & \covered     \\
      13      & \covered & \covered &          & \covered & \covered &          & \covered & \covered & \covered & \covered & \covered &          &          &          & \covered & \covered &          &          &          &          &          &           &          & \covered     \\
      14      & \covered & \covered &          & \covered & \covered &          & \covered &          & \covered &          & \covered & \covered &          & \covered &          &          &          &          &          &          &          &           &          & \covered     \\
      15      & \covered & \covered &          & \covered & \covered & \covered & \covered &          & \covered &          & \covered &          &          &          & \covered &          & \covered & \covered &          &          &          &           &          & \covered     \\
      16      & \covered & \covered &          & \covered & \covered &          & \covered & \covered & \covered &          & \covered &          &          &          & \covered &          & \covered &          & \covered & \covered &          &           &          & \covered     \\
      17      & \covered & \covered &          & \covered & \covered &          & \covered &          & \covered & \covered & \covered &          &          &          & \covered &          & \covered &          & \covered &          & \covered & \covered  &          & \covered     \\
      \bottomrule
    \end{tabular}
  }
  \end{minipage}
\end{table}

\begin{table}[htbp]
  \setlength{\tabcolsep}{4pt}
  \centering
  \begin{minipage}[t]{\linewidth}
    \centering
    \caption{Multiple Condition Coverage, per Condition}
    \label{tab:condition-coverage}
    \BigCenter{
    \begin{tabular}{c c *{3}{c} c @{\hskip .5cm} c c *{3}{c}}
      \toprule
      Decision
      & Coverage
      & \multicolumn{3}{c}{Conditions}
      &
      & Decision
      & Coverage
      & \multicolumn{3}{c}{Conditions}
      \\
      (Result)
      &
      & \multicolumn{3}{c}{(Possible Evaluations)}
      &
      & (Result)
      &
      & \multicolumn{3}{c}{(Possible Evaluations)}
      \\
      \cmidrule{1-5}
      \cmidrule{7-11}
      \\
      Line 41
      & Covered?
      & \code{i<=0}
      & \code{j<=0}
      & \code{k<0}
      &
      & Line 63
      & Covered?
      & \code{i+j<=k}
      & \code{j+k<=i}
      & \code{i+k<j} \\
      % Line 41                   % Line 63
      \cmidrule(lr){1-5}          \cmidrule(lr){7-11}
      F & \covered & F & F & F    && F & \covered & F & F & F \\
      T & \covered & F & F & T    && T & \covered & F & F & T \\
      T & \covered & F & T & F    && T &          & F & T & F \\
      T &          & F & T & T    && T &          & F & T & T \\
      T & \covered & T & F & F    && T &          & T & F & F \\
      T &          & T & F & T    && T &          & T & F & T \\
      T &          & T & T & F    && T &          & T & T & F \\
      T &          & T & T & T    && T &          & T & T & T \\
      \\
      Line 48
      & Covered?
      & \code{i == j}
      & &
      &
      & Line 74
      & Covered?
      & \code{triang>3} \\
      % Line 48                   % Line 74
      \cmidrule(lr){1-3}          \cmidrule(lr){7-9}
      T & \covered & T  &  &      && T & \covered & T  &  &  \\
      F & \covered & F  &  &      && F & \covered & F  &  &  \\
      \\
      Line 50
      & Covered?
      & \code{i==k}
      & &
      &
      & Line 76
      & Covered?
      & \code{triang==1}
      & \code{i+j>k} \\
      % Line 50                   % Line 76
      \cmidrule(lr){1-3}          \cmidrule(lr){7-10}
      T & \covered & T &  &       && F &          & F & F & \\
      F & \covered & F &  &       && F & \covered & F & T & \\
        &          &   &  &       && F & \covered & T & F & \\
        &          &   &  &       && T & \covered & T & T & \\
      \\
      Line 52
      & Covered?
      & \code{j==k}
      & &
      &
      & Line 78
      & Covered?
      & \code{triang==2}
      & \code{i+k>j} \\
      % Line 52                   Line 78
      \cmidrule(lr){1-3}          \cmidrule(lr){7-10}
      T & \covered & T  &  &      && F &          & F & F & \\
      F & \covered & F  &  &      && F & \covered & F & T & \\
        &          &    &  &      && F & \covered & T & F & \\
        &          &    &  &      && T & \covered & T & T & \\
      \\
      Line 55
      & Covered?
      & \code{triang==0}
      & &
      &
      & Line 80
      & Covered?
      & \code{triang==3}
      & \code{j+k>i} \\
      % Line 55                   Line 80
      \cmidrule(lr){1-3}          \cmidrule(lr){7-10}
      T & \covered & T &  &       && F &          & F & F & \\
      F & \covered & F &  &       && F & \covered & F & T & \\
        &          &   &  &       && F & \covered & T & F & \\
        &          &   &  &       && T & \covered & T & T & \\
      \bottomrule
    \end{tabular}
    } % Bigcenter
  \end{minipage}
\end{table}

\begin{table}[htbp]
  \setlength{\tabcolsep}{4pt}
  \centering
  \begin{minipage}[t]{\linewidth}
    \centering
    \caption{MC/DC Coverage, per Condition}
    \label{tab:MCDC-coverage}
    \BigCenter{
    \begin{tabular}{c c *{3}{c} c @{\hskip .5cm} c c *{3}{c}}
      \toprule
      Decision
      & Coverage
      & \multicolumn{3}{c}{Conditions}
      &
      & Decision
      & Coverage
      & \multicolumn{3}{c}{Conditions}
      \\
      (Result)
      &
      & \multicolumn{3}{c}{(Possible Evaluations)}
      &
      & (Result)
      &
      & \multicolumn{3}{c}{(Possible Evaluations)}
      \\
      \cmidrule{1-5}
      \cmidrule{7-11}
      \\
      Line 41
      & Covered?
      & \code{i<=0}
      & \code{j<=0}
      & \code{k<0}
      &
      & Line 63
      & Covered?
      & \code{i+j<=k}
      & \code{j+k<=i}
      & \code{i+k<j} \\
      % Line 41                   % Line 63
      \cmidrule(lr){1-5}          \cmidrule(lr){7-11}
      F & \covered & F & F & F    && F & \covered & F & F & F \\
      T & \covered & F & F & T    && T & \covered & F & F & T \\
      T & \covered & F & T & F    && T &          & F & T & F \\
      T & \covered & T & F & F    && T &          & T & F & F \\
      \\
      Line 48
      & Covered?
      & \code{i == j}
      & &
      &
      & Line 74
      & Covered?
      & \code{triang>3} \\
      % Line 48                   % Line 74
      \cmidrule(lr){1-3}          \cmidrule(lr){7-9}
      T & \covered & T  &  &      && T & \covered & T  &  &  \\
      F & \covered & F  &  &      && F & \covered & F  &  &  \\
      \\
      Line 50
      & Covered?
      & \code{i==k}
      & &
      &
      & Line 76
      & Covered?
      & \code{triang==1}
      & \code{i+j>k} \\
      % Line 50                   % Line 76
      \cmidrule(lr){1-3}          \cmidrule(lr){7-10}
      T & \covered & T &  &       && F & \covered & F & T & \\
      F & \covered & F &  &       && F & \covered & T & F & \\
        &          &   &  &       && T & \covered & T & T & \\
      \\
      Line 52
      & Covered?
      & \code{j==k}
      & &
      &
      & Line 78
      & Covered?
      & \code{triang==2}
      & \code{i+k>j} \\
      % Line 52                   Line 78
      \cmidrule(lr){1-3}          \cmidrule(lr){7-10}
      T & \covered & T  &  &      && F & \covered & F & T & \\
      F & \covered & F  &  &      && F & \covered & T & F & \\
        &          &    &  &      && T & \covered & T & T & \\
      \\
      Line 55
      & Covered?
      & \code{triang==0}
      & &
      &
      & Line 80
      & Covered?
      & \code{triang==3}
      & \code{j+k>i} \\
      % Line 55                   Line 80
      \cmidrule(lr){1-3}          \cmidrule(lr){7-10}
      T & \covered & T &  &       && F & \covered & F & T & \\
      F & \covered & F &  &       && F & \covered & T & F & \\
        &          &   &  &       && T & \covered & T & T & \\
      \bottomrule
    \end{tabular}
    } % Bigcenter
  \end{minipage}
\end{table}

\begin{figure}[htbp]
  \centering
  \begin{minipage}[t]{.8\linewidth}
    \centering
    \begin{tikzpicture}[on grid, node distance=1 and 1]
      \node[terminal] (E) {Entry};
      \node[block, below=of E] (33) {\blk{33}};
      \node[block, below=of 33] (41) {\blk{41}};
      \node[block, below left=of 41] (43) {\blk{43}};

      \node[block, below right=of 41] (47) {\blk{47}};
      \node[block, below=of 47] (48) {\blk{48}};
      \node[block, below right=of 48] (49) {\blk{49}};
      \node[block, below=2 of 48] (50) {\blk{50}};
      \node[block, below right=of 50] (51) {\blk{51}};
      \node[block, below=2 of 50] (52) {\blk{52}};
      \node[block, below right=of 52] (53) {\blk{53}};
      \node[block, below=2 of 52] (55) {\blk{55}};

      \node[block, below right=1 and 2 of 55] (63) {\blk{63}};
      \node[block, below right=of 63] (64) {\blk{64}};
      \node[block, below left=of 63] (66) {\blk{66}};

      \node[block, below left=1 and 2 of 55] (74) {\blk{74}};
      \node[block, below right=of 74] (75) {\blk{75}};
      \node[block, below left=of 74] (76) {\blk{76}};
      \node[block, below right=of 76] (77) {\blk{77}};
      \node[block, below left=of 76] (78) {\blk{78}};
      \node[block, below right=of 78] (79) {\blk{79}};
      \node[block, below left=of 78] (80) {\blk{80}};
      \node[block, below right=of 80] (81) {\blk{81}};
      \node[block, below left=of 80] (83) {\blk{83}};

      \node[joint, below=2 of 63] (J63) {};
      \node[joint, below=2 of 80] (J80) {};

      \path (E|-J80) -- +(0,-2) node[block] (86) {86};
      \node[terminal, below=of 86] (X) {Exit};

      \draw (E) -- (33);
      \draw (33) -- (41);
      \draw (41) -- node[edge label] {T} (43);
      \draw (41) -- node[edge label] {F} (47);
      \draw (47) -- (48);
      \draw (48) -- node[edge label] {T} (49);
      \draw (49) -- (50);
      \draw (48) -- node[edge label] {F} (50);
      \draw (50) -- node[edge label] {T} (51);
      \draw (51) -- (52);
      \draw (50) -- node[edge label] {F} (52);
      \draw (52) -- node[edge label] {T} (53);
      \draw (53) -- (55);
      \draw (52) -- node[edge label] {F} (55);

      \draw (55) -- node[edge label] {T} (63);
      \draw (63) -- node[edge label] {T} (64);
      \draw (64) -- (J63);
      \draw (63) -- node[edge label] {F} (66);
      \draw (66) -- (J63);

      \draw (55) -- node[edge label] {F} (74);
      \draw (74) -- node[edge label] {T} (75);
      \draw (74) -- node[edge label] {F} (76);
      \draw (76) -- node[edge label] {T} (77);
      \draw (76) -- node[edge label] {F} (78);
      \draw (78) -- node[edge label] {T} (79);
      \draw (78) -- node[edge label] {F} (80);
      \draw (80) -- node[edge label] {T} (81);
      \draw (80) -- node[edge label] {F} (83);

      \draw (83) -- (J80);
      \draw (81) -- (J80);
      \draw (79) to[bend left] (J80);
      \draw (77) to[bend left] (J80);
      \draw (75) to[bend left] (J80);

      \draw (J63) to[bend left] (86);
      \draw (J80) to[bend right=15] (86);
      \draw (43) -- ++(-4.8,0) |- (86);

      \draw (86) -- (X);
    \end{tikzpicture}
    \caption{CFG for \code{tritype.c}}
    \label{fig:cfg}
  \end{minipage}
\end{figure}

\end{document}

% vim: nowrap:guioptions+=b
