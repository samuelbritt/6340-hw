\documentclass{article}

\usepackage[T1]{fontenc}
\usepackage[osf]{mathpazo}
\usepackage{microtype}

\usepackage{amsmath}
\usepackage{amssymb}
\usepackage{amsfonts}
\usepackage{amsthm}
\usepackage{mathtools}
\usepackage{bm}
\usepackage{xcolor}
\usepackage{graphicx}
\usepackage[pdftex]{lscape}
\usepackage{lipsum}
\usepackage{tikz}
\usepackage{enumitem}
\usepackage[alsoload=binary]{siunitx}
\usepackage{booktabs}
\usepackage{multirow}
\usepackage{tabularx}
\usepackage{cancel}
\usepackage{fancyvrb}
\usepackage[section]{placeins}
\usepackage{flafter}
\usepackage{pdfpages}
\usepackage[hmargin=1.5in,vmargin=1in]{geometry}
\usepackage{url}
\usepackage{hyperref}

% algorithms
\let\oldgets\gets
\usepackage{clrscode3e}
\let\gets\oldgets
\usepackage{listings}

% no section numbers
\setcounter{secnumdepth}{-2}

% a bit more compact, and fixes spacing issues
\let\originalleft\left
\let\originalright\right
\renewcommand{\l}{\mathopen{}\mathclose\bgroup\originalleft}
\renewcommand{\r}{\aftergroup\egroup\originalright}

% commands
\renewcommand\vec[1]{\bm{#1}}
\renewcommand\epsilon{\varepsilon}
\newcommand\bit{\ensuremath{\set{0,1}}}
\newcommand\abs[1]{\l\vert #1 \r\vert}
\newcommand\set[1]{\ensuremath{\l\{#1\r\}}}
% redefine `\to` to use tikz arrows
\renewcommand\to{%
  \mathbin{\tikz[baseline=-.7ex] \draw[->] (0,0) -- +(.4,0);}%
}

% red todos
\usepackage{marginnote}
\usepackage{xifthen}% provides \isempty test
\renewcommand\marginfont{\color{red}\footnotesize}

% Arg 1: vertical shift, Arg 2: content
\newcommand\Todo[2][]{{\color{red}$^\dagger$}\marginnote{$^\dagger$#2}[#1]}

% wrapper for \Todo: can leave content and/or shift empty
\newcommand{\todo}[2][0pt]
  {%
    \ifthenelse{ \isempty{#2} }%
      {%
        \Todo[#1]{\textsc{todo}}%
      }%
      {%
        \Todo[#1]{#2}%
      }%
  }

\usetikzlibrary
{
  positioning,
  calc,
  intersections,
  backgrounds,
  arrows,
  shapes.misc,
  shapes.geometric,
}

\tikzset
  {
    ->,
    >=latex',
    shorten >=.5pt,
    node distance=.8cm and .8cm,
    fork/.style= {node distance=.8cm and #1},
    wide fork/.style={fork=1.6cm},
    block/.style=
      {
        draw,
        inner sep=.3em,
        rounded rectangle,
        text height=1.3ex,
        text depth=.2ex,
        minimum size=3.6ex,
      },
    terminal/.style=
      {
        block,
        rectangle,
        text height=1.5ex,
        text depth=.25ex,
      },
    joint/.style={fill=black,circle,inner sep=0,minimum size=5pt},
    region/.style=
      {
        draw,
        trapezium,
      },
    selected/.style={fill=gray!30},
    edge label/.style=
      {
        auto=false,
        anchor=mid,
        fill=white,
        font=\footnotesize,
        inner sep=.25ex,
      },
    data edge/.style={red},
    call edge/.style={dashed},
    param edge/.style={blue},
    trans edge/.style={green!50!black},
    inline center/.style={baseline=-.6ex},
  }

\newcommand\SampleEdge[1]{%
  \tikz[inline center] \draw[#1] (0,0) -- (.7,0);%
}

% table customizaton
\newcommand\newrow{\\\addlinespace}

% center columns
\newcolumntype{M}[1]{>{\centering\arraybackslash}m{#1}}

% Centers giant figures, tables, etc, on the page, even within nested
% enumerates/itemize environments
\makeatletter
\newcommand\BigCenter[1]{\par\noindent\hspace*{-\@totalleftmargin}\makebox[\textwidth]{#1}}
\makeatother

% highlight
\newcommand\hi[1]{\textcolor{red}{#1}}

% data analysis
\newcommand\bottom{\ensuremath{\perp}}
\newcommand\union{\bigcup}
\newcommand\Kill{\texttt{KILL}}
\newcommand\Gen{\texttt{GEN}}
\newcommand\codefamily{\ttfamily}  % use in tables
\newcommand\code[1]{\text{\codefamily #1}}
\newcommand\blk[1]{\text{#1}}
\DeclareMathOperator\pdom{pdom}
\DeclareMathOperator\dom{dom}

\begin{document}
\thispagestyle{empty}

\includepdf
  [
    pages=1,
    pagecommand=
      {
        \begin{tikzpicture}[remember picture, overlay]
          \node[text width=6.3in, text height=2.6in]
          at (current page.north east)
          {\large
            Sam Britt, Shriram Swaminathan,\\[1.5em]
            and Sivaramachandran Ganesan
          };
        \end{tikzpicture}
      }
  ]
  {Assignment8ProblemSet5}

\clearpage
\pagenumbering{arabic}

\noindent
Sam Britt                \hfill CS 6340      \\
Shriram Swaminathan      \hfill Assignment 8 \\
Sivaramachandran Ganesan \hfill Mar. 12, 2013

The CFG for \verb|tritype.c|:

\begin{center}
  \begin{tikzpicture}[on grid, node distance=1 and 1]
    \node[terminal] (E) {Entry};
    \node[block, below=of E] (33) {\blk{33}};
    \node[block, below=of 33] (41) {\blk{41}};
    \node[block, below left=of 41] (43) {\blk{43}};

    \node[block, below right=of 41] (47) {\blk{47}};
    \node[block, below=of 47] (48) {\blk{48}};
    \node[block, below right=of 48] (49) {\blk{49}};
    \node[block, below=2 of 48] (50) {\blk{50}};
    \node[block, below right=of 50] (51) {\blk{51}};
    \node[block, below=2 of 50] (52) {\blk{52}};
    \node[block, below right=of 52] (53) {\blk{53}};
    \node[block, below=2 of 52] (55) {\blk{55}};

    \node[block, below right=1 and 2 of 55] (63) {\blk{63}};
    \node[block, below right=of 63] (64) {\blk{64}};
    \node[block, below left=of 63] (66) {\blk{66}};

    \node[block, below left=1 and 2 of 55] (74) {\blk{74}};
    \node[block, below right=of 74] (75) {\blk{75}};
    \node[block, below left=of 74] (76) {\blk{76}};
    \node[block, below right=of 76] (77) {\blk{77}};
    \node[block, below left=of 76] (78) {\blk{78}};
    \node[block, below right=of 78] (79) {\blk{79}};
    \node[block, below left=of 78] (80) {\blk{80}};
    \node[block, below right=of 80] (81) {\blk{81}};
    \node[block, below left=of 80] (83) {\blk{83}};

    \node[joint, below=2 of 63] (J63) {};
    \node[joint, below=2 of 80] (J80) {};

    \path (E|-J80) -- +(0,-2) node[block] (86) {86};
    \node[terminal, below=of 86] (X) {Exit};

    \draw (E) -- (33);
    \draw (33) -- (41);
    \draw (41) -- node[edge label] {T} (43);
    \draw (41) -- node[edge label] {F} (47);
    \draw (47) -- (48);
    \draw (48) -- node[edge label] {T} (49);
    \draw (49) -- (50);
    \draw (48) -- node[edge label] {F} (50);
    \draw (50) -- node[edge label] {T} (51);
    \draw (51) -- (52);
    \draw (50) -- node[edge label] {F} (52);
    \draw (52) -- node[edge label] {T} (53);
    \draw (53) -- (55);
    \draw (52) -- node[edge label] {F} (55);

    \draw (55) -- node[edge label] {T} (63);
    \draw (63) -- node[edge label] {T} (64);
    \draw (64) -- (J63);
    \draw (63) -- node[edge label] {F} (66);
    \draw (66) -- (J63);

    \draw (55) -- node[edge label] {F} (74);
    \draw (74) -- node[edge label] {T} (75);
    \draw (74) -- node[edge label] {F} (76);
    \draw (76) -- node[edge label] {T} (77);
    \draw (76) -- node[edge label] {F} (78);
    \draw (78) -- node[edge label] {T} (79);
    \draw (78) -- node[edge label] {F} (80);
    \draw (80) -- node[edge label] {T} (81);
    \draw (80) -- node[edge label] {F} (83);

    \draw (83) -- (J80);
    \draw (81) -- (J80);
    \draw (79) to[bend left] (J80);
    \draw (77) to[bend left] (J80);
    \draw (75) to[bend left] (J80);

    \draw (J63) to[bend left] (86);
    \draw (J80) to[bend right=15] (86);
    \draw (43) -- ++(-4.8,0) |- (86);

    \draw (86) -- (X);
  \end{tikzpicture}
\end{center}

\end{document}

% vim: nowrap:guioptions+=b
